\documentclass[draft]{memoir}
\usepackage{times}
\stockletter
\pageletter
\pagestyle{ruled}
\setsecnumdepth{none}


\title{The Constitution, By-Laws, and Rules of the North County CYC Athletic Association}
\date{January 2018}
\author{}

\begin{document}

\pagenumbering{gobble}
\maketitle

\frontmatter
\tableofcontents
\mainmatter
\chapter{Foreword}
The North County District was organized to provide a coordinated sports program for the youth of the Member Athletic Associations.  It is important that every Spiritual Director, Parish Representative, Sport Coordinator, Manager and Coach take time to review all the rules and regulations under this cover.  It is our hope that with your complete and necessary cooperation, this organization will create a better understanding of what we are trying to accomplish in servicing the youth within the North County District.  With this in mind, we hope to make every effort to continue to serve God's children in the true tradition of His will.
\plainbreak{4}
This revision of the North County CYC Constitution, By-Laws and Rules is in effect as of January 2018.  
Read all of these rules very carefully.  
\plainbreak{4}
Be sure to read ALL of the data that is provided in the packets distributed at the Sport Kick-off Meeting:  the Information for Managers (Guidelines), the Official Sport Rulebook, and our Supplemental Rules.
\plainbreak{4}
KNOW THE RULES AND TEACH THEM TO YOUR PLAYERS \& PARENTS

\chapter{Constitution}
In compliance with Article III, Section 1, of The Official Constitution and By-Laws of The St. Louis Archdiocesan CYC Athletic Assocation, The North County District CYC hereby issues its Constitution, By-Laws, and Rules.

\section[Article I: Name]{Article I: Name}
This organization shall be known as the North County District CYC Athletic Association, (“NCCYC”) associated with the Catholic Youth Council (CYC) of the Archdiocese of St Louis.

\section[Article II: Purpose]{Article II: Purpose}
The general aim of the Catholic Youth Council is to help a boy or girl become an ideal youth, firmly loyal to God, church, community and country; physically fit, mentally alert and morally sound.  Athletics, properly regulated, play a very important role in the development of youth.  The Constitution’s Articles, By-Laws and Rules and Procedures are a sincere effort to promote good sportsmanship, respect for the rules and self-discipline amongst the contestants.  With the help of God, may it serve as a medium for Catholic Action.  We ask the blessing of God upon this phase of the Catholic Youth Council.

In keeping with the spirit of the program, a pre-game prayer involving both teams will be offered before all games. The home team will be responsible for leading the prayer.

\section[Article III: Policies]{Article III: Policies}
\subsection{Section 1: General Policy}
The duly elected and/or appointed Executive Board of the District shall have express authority to adopt, interpret and enforce rules and penalties consistent with this Constitution’s Articles, By-Laws and/or Rules and Procedures.

\subsection{Section 2: Managers and Coaches}
Managers and coaches, as leaders, have the responsibility of setting good example for their youths to follow.  They  shall guide their youths in accordance with Catholic principles and the purpose of the CYC Athletic Association  which is primarily dedicated to the formation of character and leadership.

\section{Article IV: Organization}
\subsection{Section 1: Executive Board}
The Executive Board shall consist of one (1) lay representative from each member athletic association, all active Past Chairpersons, Sport Officials-in-Chief, Chairperson, Vice Chairperson, Secretary, Treasurer, Sergeant at Arms, Special Advisor(s), Equipment Coordinator, and Webmaster.  Duties of the Executive Board are described in the Special Rules and Procedures, Rule ”H”.

\subsection{Section 2: Inauguration of Officers/Term of Office}
The Chairperson, Vice Chairperson, Secretary, Treasurer, Sergeant at Arms, Special Advisor(s) term of office shall run from January 1st through December 31st.  

\subsection{Section 3: Executive Officer Positions}
The Chairperson and Vice Chairperson positions are filled by an individual interested/capable of filling the role. The Chairperson and Vice Chairperson are voted into office by a majority vote of the Executive Board at the December Executive Board meeting.

The Chairperson is voted into office during an even numbered year, while the Vice Chairperson is voted into office during an odd numbered year. There are no “term-limits”, however positions will be available and voted on every two years of their respective even/odd years.

\subsection{Section 4: League Coordinator}
The Chairperson shall appoint, from the ranks of the individual parish representatives, League (Page) Coordinators for the sport in season.

\subsection{Section 5: Salary}
No member of the Executive Board shall receive any compensation from the North County District CYC Athletic Association for his/her services as board members.

\subsection{Section 6: Financial Affairs}
The financial affairs of the North County District CYC Athletic Association are the sole responsibility of the organization, with all income and disbursements being processed directly through the officers of the Executive Board.  The District will abide by the financial/procedural requirements as set forth by the Archdiocesan Financial Office.

All invoices and bills issued by the NCCYC to its member parishes shall be sent to the parish representative and the sports association treasurer and shall be due and payable within 30 days of receipt.  The receipt date shall be defined as the date mailed with the US Postal Service or sent by other electronic method to the member parish’s treasurer.  Payment shall be considered received when it is received by the Treasurer.  The parish representative will be copied on all correspondence to assist with ensuring the payment of the bill.

In the event a member parish fails to comply with this requirement, then the parish will not be in good standing with the NCCYC and will not be entitled to certain rights and privileges, specifically, representing NCCYC in the Archdiocesan playoffs and receiving trophies as a league winner.

The Chairman or Treasurer will then promptly send a communication to the parish representative and treasurer for that parish advising them they are not in good standing with the district and they have fifteen days to correct this violation.

In the event the parish fails to cure the violation within the specified time frame the member parish will not be in good standing until bill(s) are paid in full.

Should any portion of the invoice be disputed by said athletic association, the undisputed portion is to be paid in full immediately while the disputed amount/issue is under investigation. 

Exceptions:

The chairman shall grant a waiver to this rule in the following circumstances within his/her discretion:
\begin{itemize}[\textbullet]
    \item The balance due is less than \$100
    \item The parish protests the invoice, has a reasonable basis for its protest, and advises the district in writing of its protest within the 14-day deadline with resolution within 30 days. If the protest is not resolved by the district within this 30-day period, the parish's dispute/claim is honored.
\end{itemize}

The Treasurer will maintain a tracking of when its member parishes are not in good standing.  

\subsection{Section 7: Meetings}
\begin{enumerate}[A.]
    \item Executive Board:  The Executive Board shall meet at least monthly at alternating parishes.  A quorum of six (6)
    Parish Representatives within the District must be in attendance to render a valid decision, but a quorum need 
    not be present to conduct a meeting.
    \item Board of Control:  The Board of Control will be convened upon the request of the Chairperson or at the request 
    of any active Board of Control member with the current District Chairperson as the Board of Control Chairperson.
    \item Screening Meeting:  A meeting held to place the teams in the appropriate leagues.  Emphasis shall be based upon 
    the past season’s record and the consensus of the personnel in attendance.
    \item Roster Review:  A meeting to review rosters for the purpose of verifying the criteria regulatory for compliance
	Rosters are to be marked through for players not in compliance with the criteria specified.
    \item Kick-off Meeting:  A general meeting where team managers and coaches receive their schedules, rosters, cover
    rules and general information (scorecards, manager information, constitution, etc.).
\end{enumerate}

\subsection{Section 8: Executive Board Responsibilities}
The Executive Board shall have general charge of the affairs, funds, policies, and property of the NCCYC.

\subsection{Section 9: Dual Responsibilities}
Dual offices by the Executive board members are strongly discouraged, unless approved by the Chairperson.

\subsection{Section 10: Board of Control}
\begin{enumerate}[A.]
    \item The Board of Control consists of all persons who have served as Chairperson and the current District 
    Chairperson and Vice Chairperson.
    \item The current District Chairperson is the Chairperson of the Board of Control.
    \item Responsibilities of the Board of Control:
    \begin{enumerate}[1.]
        \item Evaluates and either accepts or rejects nominees for the Chairperson and Vice Chairperson.
        \item Shall be used as a source of advice or advisors to the current District Chairperson, Vice Chairperson. 
        \item In issues where the Board of Control feels the current Chairperson or Vice Chairperson are remiss in their duties, they will investigate and report the findings to the Executive Board.
        \item Responsible for the updating of the District Constitution’s Articles, By-laws and Rules and Procedures.
        \item Obtain permits from surrounding municipalities/school districts for the use of fields and gyms.
        \item Fill a temporary vacancy due to a sudden/unexpected departure of an Executive Officer until the parish fills the vacancy within a 60-day period
    \end{enumerate}
\end{enumerate}

\subsection{Section 11: Procedures}
The procedures of this organization shall be as described in "Robert's Rules of Order".  Where these rules conflict with this Constitution’s Articles, By-Laws and Rules and Procedures, the latter shall govern.
    
\subsection{Section 12: Voting Eligibility}
\begin{enumerate}[1.]
    \item A quorum of parish representatives must be present to present motions and vote.
    \item Said quorum shall be a minimum of five (5) parish representatives. Parish Representative is identified as the person speaking and voting on the behalf of the member parish.
    \item Parish Representative is defined as the person speaking and voting on the behalf of the member parish.
    \item Past Chairpersons in attendance.
    \item Sport OIC, only for the sport in-season.
    \item District chairman, only in the case of tie.
\end{enumerate}

\section{Article V: Membership}
\subsection{Section 1: Organized Divisions}
To determine the player's division for Soccer/Volleyball, use the sport year (the calendar year in which the sport year will be played) and subtract the number in the SV column.  Enter the resultant in the Year of Birth column for that number in the SV column.   Do this for all divisions.  For Basketball, Baseball, and Softball, follow the same process, except use the number in column BBS.  Match the player's Year of Birth with the year entered in the Year of Birth column.  If the player was born on or after July 31st, that is his/her correct division.

\begin{center}
    \begin{tabular}{|c|c|c|c|c|c|}
        \hline
         & & \multicolumn{2}{c|}{Soccer} & \multicolumn{2}{c|}{Basketball} \\
         & & \multicolumn{2}{c|}{Volleyball} & \multicolumn{2}{c|}{Baseball \& Softball} \\
         \hline
         Division & Grade & SV & Year of Birth & BBS & Year of Birth \\
         \hline
         Atom & 3 & 10 & & 11 & \\
         \hline
         Bantam & 4 & 11 & & 12 & \\
         \hline
         Midget & 5 & 12 & & 13 & \\
         \hline
         Crusader & 6 & 13 & & 14 & \\
         \hline
         Cadet & 7 & 14 & & 15 & \\
         \hline
         Intermediate & 8 & 15 & & 16 & \\
         \hline
         Juvenile & 9/10 & 17 & & 18 & \\
         \hline
         Junior & 11/12 & 19 & & 20 & \\
         \hline
    \end{tabular}
\end{center}

NOTE:  If the player’s birthdate exceeds July 31st of the next older birth year, he/she must play in that older division. (Each division has a range of 23 months.)

NOTE:  A player must be in the grade designated for the team on which he/she is participating.  (Exception: Baseball, Softball players who have just completed their senior year of high school may participate in the Junior Division.)

In the Intermediate Division and below, the Executive Board has the right, given supporting documentation, to allow a player who has scholastically advanced to a higher grade, to play in a grade lower, provided that player’s age falls within the requested division.

\section{Article VI: Playing Rules}
\subsection{Section 1: Rulebooks}
The playing rules, the Official Rulebooks, as issued by the Archdiocesan CYC, are based on those published by:
\begin{center}
    \begin{tabular}{l l}
        Soccer & United States Soccer Federation \\
        Basketball & National Federation Official Basketball \\
        Baseball & National Baseball Congress of America \\
        Softball & Amateur Softball Association of America \\
        Volleyball & United States Volleyball Association \\
    \end{tabular}
\end{center}
except as modified by the North County CYC Athletic Association, in supplemental rules published prior to the start of each sport season. (See rule XII for additional information.)

\subsection{Section 2: Severe Weather Alert}
In the event of severe weather, or a warning (siren), an official shall immediately stop said game at the completion of the play in progress.  Play shall be resumed from the point of suspension, providing the game has a bearing on the final league standings. A restart of the contest shall be in accordance with the playing rules for that sport. 

In Baseball and Softball, when the heat index is reported to be 103 degrees (or greater) during the game, by the local television or radio stations, the game times are to be shortened by one-half hour. Start times of succeeding games do not change, therefore creating a larger “break time” for the officials. This may only affect the mid to late day games

\section{Article VII: Protests}
\subsection{Section 1: How and When}
All playing rule protests must be made in writing and filed with the Chairperson, and must be accompanied with a protest fee of \$25.00 by check from the athletic association (NO PERSONAL CHECKS).  Only managers and representatives (coach) designated prior to the game can make an official protest.  This protest must be approved by the Parish Representative or his/her alternate.  The manager/coach protesting an infraction of the playing rules must make an announcement to the appointed official as follows:
\begin{center}
    \begin{tabular}{l l}
        Baseball & Before the next pitch or play (See rulebook for exact details) \\
        Softball & Before the next pitch, legal or illegal (See rulebook for exact details) \\
        Volleyball & At the time of the infraction, before play is resumed \\
        Basketball & At the first stoppage of play after the infraction \\
        Soccer & At the first opportunity when play has stopped \\
    \end{tabular}
\end{center}

In all sports, a brief description of the point of protest must be noted on the scorecard at the time of the alleged infraction.  Failure of the official to notify the opposing manager does not negate a protest.   This does not include eligibility (see Section 3).  If the protest is overruled, the protest fee is forfeited. If the protest is upheld, the protest fee shall be returned. A protest concerning an interpretation of a rule, governing fact of play, must be filed or postmarked no later than 72 hours following the game, (excluding Saturdays, Sundays, and Holidays) along with a \$25.00 check with the District Chairperson or Vice Chairperson 

\subsection{Section 2: Decision}
On point-of-fact, with play going on, the decision of the official shall be final.  On questions of interpretation of a rule, a protest may be announced, and noted on the scorecard, but the official's decision MUST be accepted while the game is in progress.

\subsection{Section 3: Eligibility}
The eligibility of a player(s) may be protested by letter directly with the Chairperson/ Vice Chairperson not later than the ninth (9th) day following the alleged violation or mailing a protest letter which must be postmarked no later than the ninth (9) day following the alleged violation, or within 72 hours if the game is the last one scheduled for the alleged violating team.  Should an ineligible/illegal player be accused, the Chairperson will immediately notify the alleged offending team’s Parish Representative. Confirmation of ineligibility will result in the possible forfeiture of any subsequent game(s) in which the said player(s) participate, as well as the forfeiture of prior game(s) played, which may include the offending team’s entire sport schedule.

\subsection{Section 4: Protest Hearings}
\begin{enumerate}[A.]
    \item The burden of proof is on the protesting team.
    \item All teams affected by the protest shall be given the opportunity to be present at the meeting.
    \item The Protest Hearing Board is chaired by the Sport OIC.
    \item Official(s) shall be present if the Protest Hearing Board Chairperson deems it necessary.
    \item Notice of the ruling shall be given to all teams in the affected league.  No games will be forfeited unless a team legally protests and proves the ineligibility of a player, not to include the Vice Chairperson’s actions in reviewing the scorecards.
    \item PENALTY: Games/matches covered in a 30-day period prior to the date of protest or date of District involvement in which ineligible players are used may be award-ed to the opponent or opponents of the team violating eligibility rules. The date of the District involvement should be filed in writing with the District Chairperson or Sport Chairperson. Managers and players of the offending teams will be reprimanded and/or suspended as the Hearing Board deems necessary.
\end{enumerate}

\subsection{Section 5: All Appeals}
\begin{enumerate}[A.]
    \item An appeal fee of \$35.00, a check from the athletic association, must accompany any appeal of a Protest Hearing decision.
    \item Appeals must be made in writing by the Parish Representative to the District Chairperson.  The appeal will be  heard by the Executive Board.  Appeals must be postmarked or filed with the District Chairperson/Vice Chairperson by the sixth (6) day after the date of the Protest Hearing.
    \item A second and final appeal of \$50.00 may be made to the Archdiocesan CYC Executive Athletic Committee, by the sixth (6) day after the date of the District Appeal Hearing.
    \item All appeals must be based on the same infraction or rule violation(s) as stated in the original protest.
    \item Any appeal fee tendered will be returned if said appeal is upheld.
\end{enumerate}

\subsection{Section 6: Informal Complaint}
Executive Board members have the right to monitor, act on and if need be ENFORCE eligibility without a FORMAL PROTEST being lodged by a team in that league, as specified in Section 3 of this Article.  This may be accomplished by a verbal complaint if presented at an Executive Board Meeting or a written letter to the Chairperson/Vice Chairperson, he/she will immediately notify the alleged offending team's Parish Representative.  Failure to resolve the reason for the alleged violation will result in the forfeiture of any subsequent game(s) in which said player(s) participate.

\section{Article VIII: Proviso}
These By-Laws and Rules should remain consistent with the Archdiocesan Constitution and By-Laws, however the North County District may, for a peculiar circumstance within district play, change/waive Archdiocesan Constitution and By-Laws to keep within the spirit of this constitution

\section{Article IX: Changes}
\subsection{Section 1: Constitution}
Any Article in this Constitution may be added, amended, or repealed by a written proposal at any time and a two-thirds (2/3) favorable vote of the Executive Board, presented at two (2) consecutive meetings where a quorum is present.  The written copy of this Constitution will be updated each year following our Constitution Meeting which will follow the annual Archdiocesan Constitution Meeting.  

\subsection{Section 2: By-Laws}
Any By-Law may be added, repealed or amended by a majority vote of the Executive Board presented at two (2) consecutive meetings when a quorum is present.

\subsection{Section 3: North County Rules \& Procedures}
A Rule or Procedure may be added, repealed or amended by a majority vote of the Executive Board at any meeting when a quorum is present.

\chapter{By-Laws}
\section{By-Law I: Meetings}
\subsection{Section 1: Protest Hearing}
\begin{enumerate}[A.]
    \item Established by Article VII, Section 4. of the Constitution to hear protests.
    \item A protest will first be reviewed by the District Chairperson or Vice Chairperson to determine if it was filed in accordance with Article VII, Section 1 of the Constitution.  If a protest is filed, they will then contact the Official-in-Chief to determine that the protest is on a rule infraction and not on a judgment call.
    
    If the facts of the rule(s), the game situation, and the game official’s/manager’s reports are such that the results of a protest are obvious, the Executive Officers, to include the Official-in-Chief, may rule accordingly, negating the need for a protest.  Said decision may be appealed to a Protest Appeal Board by the offended party.
    \item The Executive Board will serve as the Protest Board. It will convene at its monthly Executive Board meeting and will hear protests in accordance with the following guidelines. A “Special” Executive Board Meeting may be called for by the District Chairperson or Vice Chairperson if a timely expedition is required. Members will be given at least forty-eight (48) hours’ notice.
\end{enumerate}

\subsubsection{Protest Board Guidelines}
\begin{enumerate}[A.]
    \item The Protest Board Chairperson will be the District Chairperson or Vice Chairperson.
    \item Membership

    Each active athletic association in the District will be allowed one (1) member who is allowed to vote.  The Parish Representative from each athletic association has the responsibility to attend.  In his/her place, another active member of their parish may attend.  ONLY with prior written permission given by the District Chairperson, may observers be present.
    \item Non-voting Members 

	The Official-in-Chief should be in attendance upon the request of the Chairperson.  NOTE:  The OIC may participate with the approval of the Chairperson. 

	The Secretary (or a designated alternate) should be in attendance for the purpose of taking, writing and distributing the meeting minutes.  
    \item Meeting   

	The time and location shall be determined by the monthly Executive Board Meeting rotation as identified in Article VII, Section 1 of the Constitution.  A quorum of four (4) is required.  NOTE:  A Protest Board MUST be called within ten (10) days after the receipt of a valid protest.  A “Special” Executive Board Meeting may be called for by the District Chairperson or Vice Chairperson if a timely expedition is required. Members will be given at least forty-eight (48) hours’ notice.

    \item Order of Business
	
       Chairperson:
       Reads the manager's letter of protest.
	
	Chairperson:
       Reads the official's report

       Officials:  
       Official(s), if requested, shall appear individually.  Members may ask questions, when recognized by the 
       Chairperson, pertaining strictly to the protest.

	Chairperson:
       Has full control of the discussion by setting a time limit on each case or on the discussion by each member. 
       He/she shall also solicit participation from all members

	Chairperson:  
       Shall control and solicit discussion as noted above.

    \item Motions

    Any member, with the exception of the protesting and opposing parishes, may place or second motions.

    Chairperson:  Votes only to break a tie.
    
    NOTE:  Voting options are:  YES, NO, ABSTAIN.
\end{enumerate}

\subsubsection{Appeal}
\begin{enumerate}[A.]
    \item Established by Article VII, Section 4 of the Constitution to hear disciplinary action appeals and protest appeals.
    \item An appeal will first be reviewed by the Chairperson or Vice Chairperson to determine if the appeal was filed in accordance with Article VII, Section 1 of the Constitution.
    \item An appeal of a protest board decision or a disciplinary action must be submitted with the offended athletic association’s approval and be accompanied by an athletic association check in the amount of \$35.00.
    \item A second and final appeal may be made to the Archdiocesan CYC Executive Sports Committee and be accompanied by an athletic association check in the amount of \$50.00.
    \item The Executive Board will convene at its monthly Executive Board meeting and function in accordance with the following guidelines.
\end{enumerate}

\subsubsection{Disciplinary Action Guidelines}
All serious misconduct offences and disciplinary actions are identified as:

\begin{center}
    \begin{tabular}{| m{7cm} | m{4cm} |}
        \hline
        Obscene language or gesture toward an opponent, official or spectator & Four Game Suspension \\
        \hline
        Racism/Bias	& One Year Suspension \\
        \hline  
        Fighting (with blows) &	One Year Suspension \\
        \hline
        Player pushing and/or shoving (no blows) & Two Game Suspension \\
        \hline
        Manager, coach, or adult spectator pushing and/or shoving & One Year Suspension \\
        \hline
        Refusing to leave the property after being ejected (park, gym, parking lot, etc.) & One Year Suspension \\
        \hline
        Suspended person present at a came or is in the facility & One Year Suspension \\
        \hline
        Throwing equipment, chair(s), or another object(s) towards an opponent, official or spectator & One Year Suspension \\
        \hline
        Throwing equipment, chair(s), or other object(s) not directed at a person & Four Game Suspension \\
        \hline
        Ejections & Four Game Suspension \\
        \hline
        Confronting an official after being ejected	& One Year Suspension \\ 
        \hline
        Obscene language (not directed towards a person) & Two Game Suspension \\
        \hline
        Other misconduct not defined elsewhere & Up to One Year Suspension as decided by District Executive Board \\
        \hline
    \end{tabular}
\end{center}

Multiple offences are subject to a lifetime suspension as deemed by the District Executive Board		
Any penalty/penalties shall be carried over to the next sport in which they participate.  This shall include 
officiating for any sport.

\subsubsection{Reports}
The District Vice Chairperson shall be responsible for recording all Disciplinary Actions for distribution at the next Executive Board meeting.  This may be a separate handout, or it may be incorporated as part of the Executive Board Minutes.

\section{By-Law II: Registration of Players, Coaches, and Managers}
\subsection{Section 1: Sports}
The North County CYC Athletic Association may provide for, but is not limited to, the following sports:  Baseball, Softball, Soccer, Volleyball, Lacrosse, Chess, Track and Field, Golf and Basketball.

\subsection{Section 2: Team Entry Form}
To enter a team in a league of the Athletic Association, each Team Entry Form, supplied by the District, must be filled out completely, containing the following information:
\begin{enumerate}[A.]
    \item Name of the parish Athletic Association
    \item The name, email address, street address and telephone number of the manager
    \item The division in which the team will participate
    \item League record from previous year, if applicable
    \item Team ranking within its own parish, in each division
    \item Request for special consideration, concerning playing days/times
    \item Request for placement within a division, A1, A2, B1, B2, etc.
    \item Identify Team as Open or Closed.  Should the parish fail to do so, the District shall declare the team to be Open.
    \item Have signature of Parish Sport Coordinator or Parish Representative
\end{enumerate}

\subsection{Section 3: Parish Activity List}
These activities must be parish-wide, a religious event or an academic function.  The District will schedule around these events when submitted.

We will ATTEMPT to schedule around Girl/Boy scout events if they are listed.  We WILL NOT RESCHEDULE around Girl/Boy Scout events if they are not listed.  An exception will be any Archdiocesan Directive. 
One copy should be submitted with the Team Entry Forms.

Below are examples of events that should be included on the Parish Activity List:

\begin{center}
    \begin{tabular}{>{\small}l >{\small}l >{\small}l >{\small}l}
        PSR & Speech Meets & Talent Shows & High School Open House \\
        First Communion & Graduation & D.A.R.E. Graduation & High School Testing \\
        Field Trips	& Plays & Confirmation & Picnic \\
        Retreats & School Camps & Missions & Festival \\
        Track Meets	& School Concerts & Father/Daughter Dance & Carnival \\
        Spelling Bee & Rehearsals & Math Bee & “Lock In” \\
        Geography Bee & Parish Socials & Mother/Daughter Dance	\\
        Dinner	Auctions & May Crowning & & \\
    \end{tabular}
\end{center}

\subsection{Section 4: Rosters}
\begin{enumerate}[A.]
    \item All players’ names (the complete first name, NO nickname, and then the last name), addresses, zip codes, their Parish of Registration, Parish of Residence, School Attending, the day (M-Monday, T-Tuesday, W-Wednesday, Th-Thursday, F-Friday) that they attend PSR, if applicable, the grade that they are currently in and the School District in which they reside, shall be entered on the roster form accordingly.  If a player is being released from another parish, the Type of Release (P – Parish, H – Handicap, OP – Open Parish), the name of the Releasing Parish shall be entered thereon.  If the player is an OPEN player, enter an O in the O field.  If the player is Closed, leave this field blank.
    \item Enter the manager’s/coaches’ names, addresses, zip codes, phone numbers, email address, Archdiocesan coach’s number in the appropriate spaces.
    \item Junior and Juvenile teams must have one (1) adult (21 or older) on the roster (manager, coach).  The roster must reflect an "A" at the end of the "Name" column to indicate that the participant is an adult.
\end{enumerate}

\subsubsection{Notes:}
\begin{enumerate}[1.]
        \item Player’s names must be entered on the roster in alphabetic order by the STREET ADDRESS at which they reside.
        \item For all fields that require a parish/school name or the school district, the corresponding abbreviation as identified in the District’s Directions for Completing the Team Roster Form shall be used.  If you have a school that is not identified in this directive, contact one of the executive officers for appropriate guidance.
        \item NO player shall be allowed to participate in a game/match unless his/her name appears on the official roster with the official North County CYC approval
        \item The roster MUST be present for ALL games/matches.  The manager has until the end of the game/match to produce said document(s).  Failure to do so will result in a forfeiture of that game/match.
        \item Team Names/submittals must be in the proper format; Grade – Gender – Classification (A/B) -  Parish  -  Coach Name – Open/Close
            
        Example – 8-G-A-SF-Johnson-O

        \item Rosters are to be submitted to the District Chairperson with the appropriate amount check for said team fee.
\end{enumerate}
    
\subsubsection{Roster Distribution:}
\begin{tabular}{l p{9cm}}
    Paper & Identified with Official NC CYC approval and given to the team manager at the Sport Kick-Off Meeting. \\
    Electronic & District Executive Board and Archdiocesan CYC Office \\
\end{tabular}
\plainbreak{1}
Three (3) copies of each release must be submitted at the same time the roster, but no later than the date determined/announced by the Chairperson.   One (1) copy of the release will be provided to the releasing athletic association, one (1) copy for the receiving athletic association and one (1) copy for the Vice Chairperson.

\subsection{Section 5: Roster and Release Review}
A committee chaired by the Vice Chairperson, to include members of the Executive Board, will review the team rosters and player releases for discrepancies.  Any player that is found lacking the proper information per Section 4A during this review will be scratched from the roster.  A player may be re-instated, via resubmittal of the roster. Any discrepancies found will be identified to the appropriate Parish Representative and clarified/corrected prior to the Sport Kick-off Meeting. 

Failure to submit corrected/missing information, including the Parish Move Form by the Parish Representative within 10 days after notification of said Parish Representative, will result in forfeiture of the games played until the information is received.

\subsection{Section 6: Changes}
Modifications to the official roster will be accomplished by resubmitting the roster to the Chairperson or Vice Chairperson and returned to the manager BEFORE the player is considered eligible.  The Vice Chairperson, in his/her absence, his/her chosen alternate or the Chairperson, shall record the date and time of receipt.  If the form is mailed, the postmark is considered the filed date and must be forty eight (48) hours earlier than the date of the game in which the new player(s) participate(s).  Additions must be postmarked or in the hands of the NC CYC, forty-eight (48) hours before the game is played.

The cut-off date for adding players and coaches is as follows:

\begin{center}
    \begin{tabular}{l l l}
        Sport & Division & Date \\
        \hline
        Soccer/Volleyball & All Divisions &	October 1 \\
        Basketball & All Divisions & February 1 \\
        Baseball/Softball & Intermediate and down & May 1 \\
        & Juvenile and up &	July 1 \\
    \end{tabular}
\end{center}
	 
These dates will be published in the Manager Information Sheets for each sport season.  Players who are not active team members should be deleted.  Deletions will be accepted at anytime.

Players and coaches may be added to a team after the designated cut-off date only if there is sufficient reason to justify such action (i.e., loss of person due to injury, person moved out of the parish, etc).  New residence may be added at any time.  The acceptability of such additions will be determined by the District Chairperson.

\section{By-Law III: Eligibility}
\subsection{Section 1: Player}
\subsubsection{A. Atom, Bantam, Midget, Crusader, Cadet and Intermediate Divisions}
A player is eligible to participate in the North County CYC District Program in the above divisions in the following manner:

\begin{enumerate}[1.]
    \item The Parish of Registration or the Parish School that the player attends.  The choice is entirely up to the player.
    \item Parish of Residence  -  The Parish of Residence will apply if Parish of Registration/School Attending does not have a team in the player's division.
    
    A player who changes residence will be eligible to play for his/her previous parish for one (1) year from the date of the move.  A Parish Move Form must be filed with the District Chairperson.  (Submit with the Team Roster)

    \item Notification of a change in Parish of Registration, but not residence must be provided prior to the start of the sport season
\end{enumerate}

\subsubsection{B. Intermediate Division and Down}
If there is no program in the player's athletic association, the player may apply to their Parish Representative for assignment to another athletic association. The official District Release Form must be completed and submitted with the team Roster. The player must be assigned to a team in his/her division, or higher, based on the needs of the athletic association team and not on geographic or other considerations.  Assignment of these players is based upon the approval of the accepting athletic association.  The player(s) is/are subject to the rules of that athletic association. The purpose of this is to aid athletic associations that have a problem getting enough players and also to provide a place for players who wish to participate.

\subsubsection{C. Junior and Juvenile Divisions}
OPEN PLAY ONLY TEAMS are allowed to carry players on their rosters regardless of their participation in other programs with the exception of the Missouri State High School Activities Association (MSHSAA).  Players participating on their high school team are ineligible to play the same CYC sport at the same time.

\subsubsection{D. Excess Players}
If a parish has an excess of players, too many for one (1) team, not enough for two (2), the subject athletic associations, the Parish Representative may attempt to place the excess player(s) on another team up until the cut-off date to acquire players. After that, it will be up to the District Chairperson.  Player assignment will be in accordance with (B) above.

For  “A” Teams:  In accordance with the Archdiocesan Constitution and By-Laws, an excess of players shall be considered to be over:
\begin{center}
    \begin{tabular}{l r}
        \hline
        Soccer & 14 \\
        Volleyball & 9 \\
        Basketball & 8 \\
        Baseball & 12 \\
        Softball & 13 \\
        \hline
    \end{tabular}
\end{center}

For “B” Teams:  The minimum number shall be at the option of the releasing athletic association.  Under no circumstances will the athletic association be allowed to release a player(s) if it brings the team below the maximum allowed to start the game/match.
\begin{center}
    \begin{tabular}{l r}
        \hline
        Soccer & 11 \\
        Volleyball & 6 \\
        Basketball & 5 \\
        Baseball & 9 \\
        Softball & 10 \\
        \hline
    \end{tabular}
\end{center}

The released player(s) must follow the prescribed requirements for obtaining a release.

\subsubsection{E. Player Releases}
A player may play for another athletic association even though their own athletic association has a team in that division. 

\begin{enumerate}[1.]
    \item Releases are required for all out-of-district players, Intermediate and down.
    \item Out of district releases are allowed only with the permission of the Chairman from the releasing District.
    \item In the Juvenile and Junior Divisions, releases are not required for players participating in district games.  However, in order for any of these players to participate in the Archdiocesan Championship Playoffs, releases must be presented to the District Chairperson (Required releases should be turned in with rosters.)
    \item Releases must be approved/signed by the Parish Sport Coordi­nator, Parish Representative and by the Manager(s) of any/all teams the releasing athletic association has in that division.
    \item A team that has released players to another athletic association may not accept players re­leased by another athletic association, and vice versa, (Intermediate and below) unless there are extenuating circumstances in the athletic association. This will be reviewed by the District Chairperson.
    
    If a player has been previously, legally released by his/her athletic association for a given sport, he/she will be considered eligible for release, in the subsequent sport season, to the team to which he/she was originally released to.

    \item A Closed player can be reassigned to a Closed or an Open team, regardless of the Releasing team’s classification.
    
    An Open player can only be reassigned to an Open team, regardless of what the Releasing team’s classification is. That player’s assignment can only be to a team within their district, unless there is no Open team in that district.  If there is no Open team on which the player can participate, then the Open player may cross district lines with the written consent of both District Chairpersons.
\end{enumerate}

\subsubsection{F. Limitations}
For all divisions a player may participate, simultaneously, with one CYC team (Exception: Parochial division) and one or more non-CYC teams, in the same sport, provided their North County CYC team has been declared an Open Team and is approved by their Parish Athletic Association at the Sport Screening Meeting.  All other teams will be considered Closed and any player found to be participating simultaneously with a team in the North County CYC Athletic Association and any other team in the same sport will be subject to disciplinary action by the Executive Board (EXCEPTION - Junior, Juvenile, Bantam, and Atom are considered Open.)

\subsubsection{G. Compensation}
No player shall receive any monetary compensation for his/her services rendered to a team of the North County CYC. This rule shall not prevent suitable prizes from being given to players. 

\subsubsection{H. Emergency Coach}
In the absence of the official manager/coaches, a substitute manager/coach, with or without a CYC ID Card, NOT shown on the roster, may participate in said game/match as an emergency coach. Emergency coach must have taken the Protecting God’s Children class.

\subsubsection{I. Accomodations}
The North County CYC recognizes that some children may not be capable of playing sports with children their own age.  When an athletic association can document that a handicap exists, a child may be granted permission, on a case by case basis, to play with an age group one (1) year younger.  A form must be completed by the parents and the Parish Athlet­ic Association Chairperson and submitted to the Executive Board for approval before the child may be entered onto the roster.  The player shall be placed on a B-team, if available.

\subsection{Section 2: Identification Cards}
\begin{enumerate}[A.]
    \item All managers, coaches and players must have an Official CYC Identification Card to be eligible to participate in any contest.  Non-playing manager’s/coach’s ID Cards do not expire.  Managers and coaches must have a coaches ID number as provided by Archdiocesan office.  All managers, coaches must have completed the Protecting God’s Children program, police background check, read The Code of Ethical Conduct book, sign the commitment to Ethical Conduct page.  Scorekeepers do require an ID Card if sitting on the bench.
    \item In special cases, the Parish Representative, or alternate, may request from the Chairperson, a temporary permit to allow the player, manager or coach to participate in a contest.  This permit must be requested at least two (2) days prior to any scheduled/rescheduled contest.  This permit will be good for ten (10) days, and no second permit will be issued to anyone, except under extenuating circumstances, at the discre­tion of the District Chairperson.  Temporary permits are NOT valid for the Archdiocesan Playoffs.
    \item The game officials (or District Official) will check the ID Cards and the official roster for each team A specific player card/ID check may be checked against the player if requested by either manager. Spot checks throughout the sport season may happen as well.
    \item If one (1) or more (to include the entire team) participants do not have their ID Card(s) at the start of the contest, that team will forfeit.
    \item In the event of an ejection per Rule IV, the game official shall obtain the ejected person’s ID card and, as such, it becomes part of his/her report. It remains with the North County CYC Athletic Association until the suspension is resolved.
        
    NOTE:  Failure of the official to secure the ID Card does not void the ejection or automatic suspension.  Alternate IDs (e.g., motor vehicle licenses) shall not be surrendered.
\end{enumerate}

\section{By-Law IV: Conduct}
\subsection{Section 1}
All that is dishonorable, unsportsmanlike and ungentlemanly is particularly and expressly condemned.  The player, coach, manager, spectator found guilty of violating these principles shall be removed from the game at the discretion of the official and be put on report to the North County CYC.  Additional discipline/penalties may be assessed.

\subsection{Section 2}
Any person directly or indirectly approaching a player of another team for the purpose of inducing him/her to leave that team shall be liable to punishment as deemed appropriate by the Executive Board.

\subsection{Section 3}
Rules of conduct to be adhered to by participants in all CYC sports:

\begin{enumerate}[A.]
    \item PROFANITY
    
    Will not be tolerated either on or off the court, field or in the gymnasium while under the jurisdiction of the     CYC.
    
    \item UNSPORTSMANLIKE CONDUCT 
    
    Has no place in the field of sports, particularly by Catholic participants.

    \item RESPECT FOR AUTHORITY
    
    The officials are the representatives of the District, and as such, have the authority on the field, court or           gymnasium until they leave the premises. We must insist that respect be shown to proper authority as a part of essential       training, in accordance with the purpose of the CYC.
    
    \item PUNCTUALITY
    
    Both by members of the teams and officials.  All games should start on time, in order not to cause a delay in    the start of the following game(s).

    \item DANGEROUS PLAY
    
	The idea or concept of any game is to play according to the rules and not to cause bodily harm. All players should be taught and disciplined to play the ball and not the opposing player.

    \item SMOKING, INTOXICATING BEVERAGES, CONTROLLED SUBSTANCES, AND FIRE ARMS AS DEFINED BY MISSOURI LAW
    
	Are prohibited on the bench, on the field, on the court or any playing surface.

    \item Priests and other religious persons, managers, coaches, players, scorekeepers and medical personnel are the only persons permitted on the bench.
\end{enumerate}

\subsection{Section 4}
The manager is responsible for his/her actions as well as those of his/her coaches, players and spectators.  IF YOU DO NOT MEET YOUR RESPONSIBILITIES AND DEGRADE THE CYC PROGRAM, YOU DO NOT BELONG IN THE PROGRAM.  If you have constructive criticism regarding officials, contact your Parish Representative.

\section{By-Law V: Suspensions}
\subsection{Section 1}
Any player, coach, manager or spectator is SUSPENDED INDEFINITELY from ALL ACTIVITIES AND FACILITIES under the jurisdiction of the North County CYC Athletic Association, until he/she serves out their disciplinary action, if he/she is:

\begin{enumerate}[A.]
    \item Ejected from a game (EVEN IF NOT put on report by the game/match official).
    \item Put on misconduct report by a game/match official before, during or after a game.
\end{enumerate}

\subsection{Section 2}
\begin{enumerate}[A.]
    \item Suspensions are issued in terms of games or a period of time, and may overlap into succeeding sport seasons.
    \item Suspensions will cover North County CYC League, Division Playoff and Archdiocesan Playoff games.
    \item All tournaments hosted by athletic associations within the North County District must honor our suspensions.
\end{enumerate}

\subsection{Section 3}
If any player, coach or manager participates in a game while under suspension, or a suspended spectator is present, the team will suffer a forfeit and the offender will be served with disciplinary action.

\subsection{Section 4}
This Association recognizes the suspensions of other organizations.

\section{By-Law VI: Verification of Ages}
\subsection{Section 1}
The Parish Athletic Association shall be responsible for the accuracy of information for all participants on the team roster.

\section{By-Law VII: Elimination Contests}
\subsection{Section 1}
At the end of each sport season, there will be a playoff game involving the win­ner of the "A" Leagues, if there was more than one “A” League.  The home team will be determined by the flip of a coin, pre­sided over by the game official(s).  Once a team is determined as home team, it will continue to be the home team should the game be suspended or replayed.   Overtime/extra innings are authorized.

\subsection{Section 2}
The Division Champion shall be the winner of the "A" League Playoff.  The first-place winner of each "A" League of each division will have an opportunity to represent the District in the Archdiocesan CYC Playoffs. “B” teams are not eligible for playoffs.

\subsection{Section 3}
The following procedure will be used to send two (2) teams, within the same di­vision, to the Archdiocesan Playoffs when a request is received from the Archdiocesan office.

\begin{enumerate}[A.]
    \item If there is only one (1) "A" League within the division, the league winner and the sec­ond place team will have an opportunity to represent the District.
    \item If there are two (2) “A” Leagues within the division:
    
    The "A1" league winner will play a contest against the “A2” league winner. The winner will be given the opportunity to represent the District at the Archdiocesan Playoffs and the looser will be given the opportunity to represent the District in the Wild Card spot (if the wild card spot is available to our District)

    \item If any of the above teams are unable to represent the District in the Archdiocesan Playoffs, the District Chairperson shall select a team to take their place.
\end{enumerate}

\subsection{Section 4: Trophies}
The players on all league winning teams will receive individual trophies.  A minimum of three (3) teams of the same classification is required to be eligible for trophies.

\section{By-Law: VIII: Wagering}
NO wagering of any sort will be tolerated. The violator WILL BE expelled from the Program.

\section{By-Law IX: Officials (Referees/Umpires)}
\subsection{Section 1}
The Official-in-Chief shall be nominated by an Athletic Association and must be approved by the District Chairperson.  His/her duties and term of office are covered in North County Rule "E", Section 4 "Official-In-Chief."

The Official-in-Chief may serve indefinite terms at the discernment of the District Executive Board. When a vacancy occurs, it shall be filled by a representative of participating athletic association next in line, following the rotation identified below

\begin{center}
    \begin{tabular}{ r l }
        1. & St. Sabina \\
        2. & St. Rose Philippine Duchesne \\
        3. & Blessed Theresa of Calcutta \\
        4. & St. Norbert \\
        5. & Christ, Light of Nations \\
        6. & St. Ferdinand \\
        7. & Our Lady of Guadalupe \\
        8. & St. Martin de Porres \\
        9. & St. Ann \\
        10. & St. Angela Merici \\
        11. & Sacred Heart \\
    \end{tabular}
\end{center}

\begin{enumerate}[A.]
    \item If an athletic association cannot provide a volunteer from within its own organization, it may recruit a volunteer from another athletic association. That volunteer will fill the position on the behalf of the said athletic association, not the athletic association in which he/she normally participates.
    \item The specified athletic association has sixty (60) days after receipt of written notification from the Chairperson to fill the position.
    \item If the specified athletic association fails to fill the position, the athletic association, next in the rotation, will have sixty (60) after receipt of written notification from the Chairperson to fill the position.  This cycle will continue until the position is filled.  Until the position is filled, member(s) of the Executive Board will perform  the necessary duties.
    \item The penalty for non-compliance shall be as follows:  The offending athletic association will not participate in any Archdiocesan Playoffs until the position is filled.
    \item The Executive Board may vote to give a waiver or pass over an athletic association that cannot fill a position.
    \item An athletic association may not hold more than two (2) positions at the same time.  If such a situation arises, that athletic association shall be passed over to the next athletic association specified in the rotation.
\end{enumerate}

\subsection{Section 2}
The game officials are obligated to file a written report, within forty-eight (48) hours with the OIC. The OIC shall forward that report to the District Chairperson for all ejections, misconduct and protests.

\subsection{Section 3}
It is the game official's responsibility to make sure that both managers know that a game is being played under protest.  However, failure to notify the other manager shall not void such protest.

\subsection{Section 4}
The following rules are provided for game officials:

\begin{enumerate}[A.]
    \item Excessive, unnecessary roughness shall result in ejection from the game.
    \item Depending on the circumstances, the official does NOT have to give a warning before ejecting a participant, manager, coach, scorekeeper, medical personnel, or spectator. The “warning” has been issued at the sport kickoff meeting.
    \item If, after a warning, team conduct is not satisfactory after continuing, or attempting to continue the game, the official may suspend play.  Final determination of the game will be decided by the Chairperson following the guidelines covered in the By-Laws under By-Law I.
\end{enumerate}

\section{By-Law X: Forfeiture}
\subsection{Section 1}
Any team that forfeits three (3) consecutive games or a total of four (4) league games in a season, will be dropped from the league.  All games that have been recorded will stand and all remaining games will be recorded as forfeits.  For the purpose of this rule, a forfeit is defined as any game that is not started.

\subsection{Section 2}
In the event both teams fail to appear for a game with a sufficient number of players, both teams will be charged with a chargeable forfeit and under no circumstances will the game be rescheduled.  

NOTE:  Does not apply if the game is officially postponed for some other reason.

\subsection{Section 3}
An assessment of \$50.00 per forfeit shall be billed to the athletic association at the end of the sport season.  All forfeit money will go to the Graham/LePage Scholarship Fund.  If a team notifies the League Coordinator at least seven (7) days before a game is to be forfeited, their Athletic Association will not be billed the \$50.00 forfeit fee.

\section{By-Law XI: Parish Constitution and By-Laws}
Any Athletic Association may draw up a constitution and by-laws for the management of that respective parish, not at variance with this Constitution and By-Laws or the spirit thereof, as inter­preted by the Executive Board.  This constitution and by-laws must be on file with the District Chairperson.

\section{By-Law XII: Supplemental Playing Rules and Procedures}
\subsection{Section 1}
The North County CYC Supplemental Rules and an Archdiocesan Rulebook will be available to each team manager online at NCCYC website.

\subsection{Section 2}
Any individual may present a proposed supplemental rule directly to the Executive Board for approval (at least 90 days prior to the start of the season).
 
\section{By-Law XIII: Knowledge of Constitution and By-Laws}
Each manager or coach shall be supplied with, or made available to, a copy of this Constitution, By-Laws and Rules. They shall acquaint their coaches, players/parents and spectators with the contents.  Ignorance shall not be accepted as an excuse for violation of this Con­stitution, By-Laws and Rules.

\section{By-Law XIV: Rescheduling}
\begin{enumerate}[A.]
    \item Games can be rescheduled through the appropriate League Coordinator ONLY with the approval of the District Chairperson, after the requesting team fulfills the following requirements:
    
    The request, resulting from a previously approved activity, MUST be made at least five (5) days or more prior to the scheduled game, and if approved, notification must be given to the District Rescheduler no less than four (4) days prior to the scheduled game.  If this timeframe is not met, the game will not be rescheduled, unless it has a bearing upon a first place standing.

    \item Any event not listed on the Athletic Association Parish Activity List must be approved by the District Chairperson.
    \begin{enumerate}[1.]
        \item The request must be made at least five (5) or more days prior to the scheduled game and if approved, notification must be given to the District Rescheduler no less than four (4) days prior to the rescheduled game. If this timeframe is not followed, the game will not be rescheduled unless it has a bearing upon a first place standing.
        \item It MUST be a RELIGIOUS EVENT or ACADEMIC SCHOOL FUNCTION, affecting a majority of the team’s players.
        
        NOTE:  Not to include any Scouting functions or activities NOT shown on the Athletic Association’s Parish Activity List.
        \item NO  activity will be considered if an official Athletic Association Parish Activity List was NOT submitted by the requesting Athletic Association prior to the last day to add/drop a team as designated by the District Chairperson.
    \end{enumerate}

    \item Parish Representatives and managers should be alert for rescheduled and/or unlisted activities and report them as soon as possible to aid in rescheduling.
    \item A rescheduled game will be subject to the same forfeiture rules as any regularly scheduled game.
    \item A special event, (i.e., Hazelwood School District Day at Six Flags), must be approved by the District Chairperson.
    \item The League Coordinator must give the manager or acting manager a minimum of forty-eight (48) hours notice to play a rescheduled game.
    \begin{enumerate}[1.]
        \item Rescheduled games are planned far enough in advance to give managers the opportunity to take advantage of the seven (7) and five (5) day rules.  (This may not be possible during the last two (2) weeks of any sport season.)
        \item Managers shall acknowledge notification to their League Coordinator (Chairperson or Vice Chairperson if the League Coordinator is not available).  Failure to comply shall not be an acceptable reason for additional rescheduling.
    \end{enumerate}
    \item If a manager notifies the North County CYC of a forfeiture in accordance with A.1 or B.1, and on the day of the rescheduled game, the game is cancelled, the team will receive the game back.
    \item Any rescheduled baseball, softball, volleyball or soccer game that is in conflict with a schedule golf match shall be reviewed by the Golf League Coordinator.  The golf match shall be rescheduled to avoid the conflict.
\end{enumerate}

\section{By-Law XV: Athletic Assocation Mergers}
\subsection{Section 1}
All mergers must be approved by the District Executive Board. The following requirements must be met before the Executive Board considers said merger:
\begin{enumerate}[A.]
    \item Both pastors must agree.
    \item The affected athletic associations must agree and show that both parties will participate in the governing of the program.
    \item A genuine need must be shown.  An athletic association that does not have enough players to en­ter teams in the District program will be deemed to have a genuine need.   However, it must be shown that a concerted effort was made to establish a program within that athletic association.
    \item The entire athletic program must be merged, and players from each athletic association must be integrated on all teams.
    \item Responsibility for payment of all CYC fees must be determined.
    \item Merger applications, in writing, must be received by the Executive Board 90 days prior to the proposed date of the merger.
    \item A merger may be rescinded upon written request by the pastor of either partici­pating athletic association prior to the next sport season.
    \item All district rules and disciplinary action(s) evoked on individual parishes shall apply to the merged athletic associations.
    \item Archdiocesan approval must be obtained for a merger to be valid.
\end{enumerate}

\section{By-Law XVI: Seasonal Sports Merger}
This section is blank.

\section{By-Law XVII: Proviso}
Any matter NOT covered in this Constitution, By-Laws and Rules shall be brought to the North County CYC Executive Board in a timely manner. 
   

\chapter{North County Rules and Procedures}

\section{Rule ``A'': League Schedules}
\subsection{Section 1}
\begin{enumerate}[A.]
    \item League winners are determined by the total number of points accrued.  Points are determined as follows:
    \begin{center}
        \begin{tabular}{ l r }
            Result & Points \\
            \hline
            Win & 2 \\
            Tie & 1 \\
            Loss & 0 \\
        \end{tabular}
    \end{center}
    \item “A” League teams that are tied at the end of the season must play a tie-breaking game/match. If more than two (2) teams are tied, there will be a single elimination playoff established by the Chairperson. Use of a coin toss or 	draw of lots will be used to determine the pairings.
    
    “B” League teams that are tied at the end of the season will be crowned Champions and share first place.  Each team will be given individual trophies, if applicable.   

    \item Definitions
    \begin{enumerate}[1.]
        \item Division  -  A division is a grouping of teams by age and/or talent levels.
        
        Examples - Girls Midget A (GMA), Midget B1 (GMB1), Boys Intermediate Open (BIO), Boys Intermediate A (BIA), Boys Junior (BJR).

        \item League  -  A league is any four teams assembled to play together on a schedule (page) for a sport season.  A league must have at least four (4) teams and can have one (1) or more divisions.
    \end{enumerate}
\end{enumerate}

\subsection{Section 2}
No league will be larger than ten (10) teams.  Exception:  Juvenile and Junior Divisions.

\section{Rule ``B'': Adult Participation Requirements}
\subsection{Section 1}
\begin{enumerate}[A.]
    \item For the purpose of this rule, an adult is considered to be a person who is at least 21 years of age.
    \item Each team should have at least One (1) adult meeting the coaching requirements on the roster. One (1) adult, with an ID card, must be in attendance, on the bench and be responsible for the teams’ actions during the entire game, or the game will be subject to forfeited.  It is required that a second adult, with the same, be present on the bench or sideline.
\end{enumerate}

\section{Rule ``C'': Communications}
The manager's line of communication to the North County CYC Athletic Association is through his/her Parish Representative. League business, such as scheduling, rescheduling, scoring, standing issues, etc. are to be directed to his/her League Coordinator.

\section{Rule ``D'': Program Improvement}
Constructive criticism is an asset to any program, provided it is handled by the persons in charge and with an open mind.  It is recommended that any constructive suggestions for improvement in our program be submitted to your Parish Representative, so that they may be presented at an Executive Board Meeting.  An individual submitting such criticism should have a feel for the overall situation and recommend a solution to the problem.

\section{Rule ``E'': Scorecards}
\subsection{Section 1}
After the game, an official will present the scorecard to each manager for his/her signature and approval of the final score. The official score shall be as shown on the scorecard.  All blanks shall be recorded as zero (0).  Should an error be made, and the Athletic Association desires a correction, the Parish Representative must submit an official Score Card Correction Request form to the Vice Chairperson.  The Vice Chairperson will verify said information with both managers and the League Coordinator.

\subsection{Section 2}
Comments regarding game officials should be made on an Incident and presented to your Parish Representative.

\section{Rule ``F'': Uniform Colors}
The following list contains the uniform colors assigned to each parish in the North County CYC Athletic Association.  If conflict in colors exists, the team in violation, or the Home Team, must correct the problem.  For any variation in colors listed, permission must be received from the District Executive Board.  NOTE:  If soccer or basketball teams with the same primary colors are playing each other, the Home team is responsible for resolving the conflict, i.e., supply pennies.

\begin{center}
    \begin{tabular}{l l}
        Parish & Colors \\
        \hline
        St. Ann & Navy/Gold \\
        Sacred Heart & White/Red \\
        St. Norbert & Columbia Blue/Navy Blue \\
        Christ Light of Nations & Red/Blue \\
        Blessed Teresa & Royal Blue/White \\
        St. Sabina & Maroon/Gold \\
        St Angela Merici & Gold/Green \\
        Our Lady of Guadalupe & Kelly Green/Gold \\
        St. Martin de Porres & Gold/Black \\
        St. Ferdinand & Royal Blue/Gold \\
        St. Rose & Red/Black \\
        \hline
    \end{tabular}
\end{center}

Teams playing out of color may be subject to forfeiture.
               
\section{Rule ``G'': General Information/Duties}
\subsection{Section 1: Parish Representative}
\begin{enumerate}[A.]
    \item The Parish Representative has a delicate position to maintain:
    \begin{enumerate}[1.]
        \item Represent the desires of his/her athletic association and must vote in the manner prescribed by his/her athletic association.
        \item Maintain strict enforcement of the Executive Board's policies.
        \item Assume a share of the Executive Board's duties in each sport.
        \item Inform their athletic association personnel of any new policy or policy change initiated by the Executive Board.
        \item Review rosters submitted by his/her athletic association to assure accuracy of information, especially verifying Parish of Registration and the indication of.
        \item Assure that rosters are completed and filed by the designated date with the District Vice Chairperson.
        \item Assure that Team Entry Forms properly reflect the information needed by the Screening Committee and are turned in to the Vice Chairperson when requested by the District Constitution.
        \item Instruct your athletic association on the correct procedure, whether it be rosters, roster changes, protests, request for scheduling, etc.
        \item Provide the District Vice Chairperson with a list of parish/athletic association activities prior to/at the Screening Meeting for each sport.
        \item Keep a current list of athletic association personnel and provide the names and phone numbers of the Sport Coordinators.   Coordinator information must be provided not later than:
        \begin{center}
            \begin{tabular}{l l}
                Meeting & Sport \\
                \hline
                February EBM & Baseball / Softball \\
                June EBM & Soccer / Volleyball \\
                September EBM & Basketball \\
                \hline
            \end{tabular}
        \end{center}
        \item Act as/perform the duties of a League Coordinator
        \item Perform additional tasks as required; deemed necessary through the course of business.
    \end{enumerate}
    \item The Parish Representative, in his/her absence, may designate an alternate to fulfill his/her duties, with all rights and privileges.  Notification must be provided to the Chairperson/Vice Chairperson.
\end{enumerate}

\subsection{Section 2: League Coordinator}
\begin{enumerate}[A.]
    \item Must maintain communications with managers of their assigned leagues with respect to scores and win/loss   record and cancellations. 
    \item Notify the District Vice Chairperson of the need to reschedule games for the following reasons:
    \begin{enumerate}[1.]
        \item Rainouts
        \item Incomplete games (called due to darkness, weather conditions, etc).
        \item Reasons listed in By-Law XIV.
    \end{enumerate}
    \item Attempt to attend games in his/her assigned leagues to observe the conduct of players, coaches, managers, spectators and officials, and in general, the conduct of the contest.
\end{enumerate}

\subsection{Section 3: Duties of the Officers}
\subsubsection{A. District Chairperson}
\begin{enumerate}[1.]
    \item Plan the season and plan ahead for each sport.
    \item Understand that the job is very time consuming.
    \item Keep the Parish Representatives informed of all pertinent matters.
    \item Be diplomatic at all times.
    \item May request advice on policies from the Board of Control, then enforce them to the best of your ability.
    \item Despite the trials and tribulations of the job, the Chairperson has the responsibility of setting a good example of leadership without bias.
    \item Approve expenditures and expenses, holding them to a minimum.
    \item Represent the North County District at the monthly meetings of the Archdiocesan Executive Sports Committee.
    \item Serve as Chairperson of the Board of Control.
    \item Appoint the Secretary, Treasurer, Sergeant at Arms, and Special Advisor(s) for their terms of service.
    \item Take reschedule requests from the Vice Chairperson and determine if a reschedule is to be granted.
    \item Chair the monthly District Executive Board meeting.
    \item Hold meetings with the parish sport coordinators 6 weeks prior to start of sport season, as required during the sport season.
    \item Assemble the Spirit of St. Louis Tournament rosters and submit to the CYC Office.
    \item Assemble the Spirit of St. Louis Tournament shirt order and submit to the CYC Office.
    \item Notify the District league winners and inform them about the Archdiocesan Play-off meeting.
    \item Notify the League/Page Coordinators of re-scheduled games.
    \item Perform additional tasks as required; deemed necessary through the course of business.
\end{enumerate}

\subsubsection{B. Vice Chairperson}
\begin{enumerate}[1.]
    \item Attend all meetings, such as Board of Control, Executive Board, Screening, Protest Hearing, Appeals, Roster Review, Kick-off, Play-off Meetings at the CYC Office, etc.
    \item Sign and distribute all releases to all appropriate parties.
    \item Post and maintain a tracking record of all releases.
    \item Create and distribute sport kick-off “packets”
    \item Pass reschedule requests onto Rescheduler.
    \item Collect completed scorecards from various venues
    \item Create and maintain a Soccer Yellow/Red Card Report and distribute to all personnel at the Executive Board Meetings. Provide a copy to the Secretary to serve as an attachment to the EBM minutes.
    \item Create and maintain a Basketball Technical Foul Report and distribute to all personnel at the Executive Board Meetings. Provide a copy to the Secretary to serve as an attachment to the EBM minutes.
    \item Enter game scores into standings.
    \item Create and distribute any letters required for various reasons.  Provide a copy to the Secretary to serve as an attachment to the EBM minutes.
    \item Act as the District Chairperson when required by his/her absence.
    \item Shall ensure that a current copy of our Constitution, By-Laws and Rules is available on-line for the managers, coaches, spectators, etc. review/use. 
    \item Sign and seal Rosters and distribute to all of the appropriate parties.
    \item Sign and seal Supplemental Rosters and distribute to all of the appropriate parties.
    \item Review scorecards for all discrepancies and report results to the Executive Board at the monthly meetings.
    \item Maintain a record of all incident reports and distribute to all personnel at the Executive Board Meetings.
    \item Assist the Chairperson in all matters, in any way possible.
    \item Perform additional tasks as required; deemed necessary through the course of business.
\end{enumerate}

\subsubsection{C. Secretary}
\begin{enumerate}[1.]
    \item Record the attendance and create the minutes for the Executive Board Meetings.
    \item Distribute the minutes from Executive Board Meetings to all District personnel, to include the Chairperson, Vice Chairperson, Secretary, Treasurer, Special Advisor, Parish Representatives, Sport OICs, Past Chairpersons and the Webmaster for publication on the web site.
    \item Notify all appropriate personnel of all upcoming meeting(s)
    \item Process the Graham/LePage Scholarship Award
    \begin{enumerate}[A.]
        \item Distribute the nomination forms to all the Parish Representatives.
        \item Collect, process and submit the completed forms to the nomination review committee.
        \item Distribute the committee recommendations to the Parish Representatives for voting.
        \item Collect and tabulate the votes and award the scholarships to the winners at the May EBM.
    \end{enumerate}
    \item Record changes to the District Constitution and maintain the document with all appropriate changes.
    \item Send out mailings as requested by the Vice Chairperson.
    \item Collect the Parish Sport Coordinator‘s name and phone number for each sport season and compile a coordinators list.  Distribute this list at the EBM.
    \item Assist the Chairperson and Vice Chairperson in all matters, in any way possible.
    \item Perform additional tasks as required; deemed necessary through the course of business.
\end{enumerate}    

\subsubsection{D. Treasurer}
\begin{enumerate}[1.]
    \item Issue a statement each month to the Executive Board, reflecting the financial status of our organization.
    \item Issue all checks.
    \item Keep accurate records of all income and disbursements.
    \item Provide yearly summaries of comparisons of budget vs actual expenses.
    \item Ensure that the City of Florissant fee for field usage is divided equally among all teams participating in the given sport.
    \item Perform additional tasks as required; deemed necessary through the course of business.
\end{enumerate}

\subsubsection{E. Sergeant at Arms}
This is a position appointed by the Chairperson of a Past Chairperson.
\begin{enumerate}[1.]
    \item Enforce order and decorum as appropriate at Executive Board meeting.
    \item Request any meeting member to leave if he/she is disorderly or disruptive.
    \item Understand Roberts Rules of Order and how to apply them at Executive Board meetings.
\end{enumerate}

\subsection{Section 4: Official In Chief}
\begin{enumerate}[1.]
    \item Plan and hold clinics to teach the officials the rules of the sport.
    \item Assign officials to each game.
    \item Work with the District Rescheduler and Vice Chairperson to accomplish rescheduling of officials.
    \item Maintain a record of officials, showing name, age, telephone number, parish and years of experience.
    \item Keep a record of the games each official works.
    \item Report to the Executive Board, at the regular monthly meeting, the status of the sport in progress.
    \item Track and ensure District officials (for said sport) registered and paid for the “Official’s Fee”
    \item Post scores and maintain league standings, win/loss records of each team via review of scorecards.
    \item Maintain the official's uniform and equipment.  Order (through the Chairperson and Equipment Coordinator) any additional or replacement equipment.
    \item Observe the officials on the field of play and assist in the training of officials.  Also, enhance the skills/knowledge of officials.
    \item Provide an evaluation of each official at the end of each season.
    \item Pay scales/fees should be prepared by June. (see Section 6)
\end{enumerate}

\subsection{Section 5: Special Advisors}
\begin{enumerate}[1.]
    \item Will serve at the discretion of the Chairperson, and may include multiple assistants.
    \item Will support the Chairperson, the Vice Chairperson, and Sports Coordinators in their efforts to perform their tasks as outlined in the Constitution.
    \item Will report directly to the Chairperson in all areas assigned to him/her.
    \item Will apply to the Florissant Park Department for playing field permits.
    \item Obtain Field Time from the Florissant Parks Department for rescheduled games
\end{enumerate}

\subsection{Section 6: Pay Scales}
Pay scale/fees for Officials, Gym Guards, Facility Rental and Usage are presented in June and voted on at the July District Executive Board meeting, to become effective with the Soccer/Volleyball season.

\subsection{Section 7: Tournaments}
Participation by any team of the North County CYC in any tournament, pre-season, in-season or post-season, must be on a non-interference basis.  Games of the North County District must take precedence over a tournament game.

\subsection{Section 8: Support to the Archdiocesan CYC}
\begin{enumerate}[A.]
    \item A fee for all teams eligible to participate in the Archdiocesan Championship Playoffs, for all non-Playoff teams and each golf team, with golf having Spring and Fall seasons, and each lacrosse team shall be paid to the St. Louis Archdiocesan CYC Athletic Association for each sport season.  Individual fees shall be negotiated/established for each of these categories and approved by the Executive Board.
    \item One (1) copy of all rosters, manager/coach accreditation form and schedules are to be furnished to the Archdiocesan Office.
    \item All requirements above, are to be submitted before completion of the third week of each sport season.
\end{enumerate}

\subsection{Section 9: Excess Funds}
While we budget to avoid having excess money in the treasury, if after the close of the baseball/softball season, an audit reveals that excess money does in fact exist, the excess shall be refunded or credited to member Athletic Associations in direct proportion to their assessed fees during the previous year.

\subsection{Section 10: Executive Board Identification Card}
Each member of the Executive Board shall be issued a card by the District Chairperson, for identification purposes, should the need arise. 

\subsection{Section 11: Screening Committee}
\begin{enumerate}[A.]
    \item The Screening Committee is only an advisory function to the Scheduling Committee.
    \item The Chairperson, or the Vice Chairperson may chair the Screening Meeting.
    \item The committee shall meet within 7-10 days after Team Entry Form submittal and before the last day to add/drop a team.
    \item Attendance at the Screening Meeting is limited to the Screening Committee Chairperson, members of the Scheduling Committee and representatives from each parish (sport-in-season coordinators and Parish Representatives). The Screening Committee Chairperson shall take attendance.  Failure to have any representative present means that the remainder of the Screening Committee will act in that athletic association’s behalf.
    \item Failure to have completed Team Entry Forms submitted to the pre-screening group on time, could, at the discretion of the Screening Committee, cause the teams to be screened without additional inputs from the tardy athletic association.  Completed forms shall include the signature of the Parish Representative/Sport Coordinator. Should information presented on the cards be considered misleading by the Screening Committee, they shall have the prerogative of accepting their own interpretation.  Any deliberate attempt by an athletic association, in presenting Team Entry Form data or in their oral presentation at the Screening Meeting, to circumvent the intended purpose of establishing leagues of comparable competitiveness, may result in that athletic association’s team forfeiting their rights to winning the league.  A simple majority of the Executive Board will be required to enforce this last action.  The Athletic Association screening representative(s) may also be subject to disciplinary measures by the Executive Board.
    \item Implement the following guidelines for the Screening Committee activities so as to maximize, in as expeditious a manner as possible, the process leading to the formation of leagues of comparable competitiveness:
    \begin{enumerate}[1.]
        \item Have screening committee verify data presented on the Team Entry Forms at the Screening Meeting.
        \item NO parish may be forced to place two (2) teams in the same league unless they agree and can verify that the team makeup is based on ability and not random selection.
        \item A team may not be dropped, and new team added of same within the same league or division. District will reassign the players.
        \item Any major change in team structure, before the Screening Meeting, will require a NEW TEAM ENTRY FORM, before the meeting starts.
    \end{enumerate}
    \item There will be no more than two (2) "A" leagues within any age group, either Open or Closed.  The "A1" League will be the strongest league in a given division.  Atom and Bantam Leagues do NOT have “A” Leagues.
    \item The team designation of Open or Closed shall not be changed after the Screening Meeting.
\end{enumerate}

\subsection{Section 12: Executive Board Meeting Attendance}
Anyone may attend an Executive Board meeting.  However, in order to participate, notice must be entered on the meeting agenda or the District Chairperson shall be notified 48 hours prior to the scheduled meeting.

\subsection{Section 13: Team Withdrawal}
An athletic association may withdraw a team, with no penalty, anytime prior to the last day to add/drop a team as designated by the District Chairperson.  If withdrawn anytime after this designated date, the athletic association will still be charged for the team's entry fee.


    
    
    

\backmatter

\end{document}