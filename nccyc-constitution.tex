\documentclass[10pt,oneside,draft]{memoir}
\usepackage{times}
\setsecnumdepth{none}


\title{North County CYC Athletic Association Constitution, By-Laws and Rules}
\date{January 2018}
\author{}

\begin{document}

\pagenumbering{gobble}
\maketitle
\newpage
\frontmatter
\tableofcontents
\newpage

\chapter{Foreword}
The North County District was organized to provide a coordinated sports program for the youth of the Member Athletic Associations.  It is important that every Spiritual Director, Parish Representative, Sport Coordinator, Manager and Coach take time to review all the rules and regulations under this cover.  It is our hope that with your complete and necessary cooperation, this organization will create a better understanding of what we are trying to accomplish in servicing the youth within the North County District.  With this in mind, we hope to make every effort to continue to serve God's children in the true tradition of His will.

This revision of the North County CYC Constitution, By-Laws and Rules is in effect as of January 2018.  
Read all of these rules very carefully.  

Be sure to read ALL of the data that is provided in the packets distributed at the Sport Kick-off Meeting:  the Information for Managers (Guidelines), the Official Sport Rulebook, and our Supplemental Rules.

KNOW THE RULES AND TEACH THEM TO YOUR PLAYERS \& PARENTS
\newpage
\mainmatter
\part{Constitution}
In compliance with Article III, Section 1, of The Official Constitution and By-Laws of The St. Louis Archdiocesan CYC Athletic Assocation, The North County District CYC hereby issues its Constitution, By-Laws, and Rules.

\section{Article I: Name}
This organization shall be known as the North County District CYC Athletic Association, (“NCCYC”) associated with the Catholic Youth Council (CYC) of the Archdiocese of St Louis.

\section{Article II: Purpose}
The general aim of the Catholic Youth Council is to help a boy or girl become an ideal youth, firmly loyal to God, church, community and country; physically fit, mentally alert and morally sound.  Athletics, properly regulated, play a very important role in the development of youth.  The Constitution’s Articles, By-Laws and Rules and Procedures are a sincere effort to promote good sportsmanship, respect for the rules and self-discipline amongst the contestants.  With the help of God, may it serve as a medium for Catholic Action.  We ask the blessing of God upon this phase of the Catholic Youth Council.

In keeping with the spirit of the program, a pre-game prayer involving both teams will be offered before all games. The home team will be responsible for leading the prayer.

\section{Article III: Policies}
\subsection{Section 1: General Policy}
The duly elected and/or appointed Executive Board of the District shall have express authority to adopt, interpret and enforce rules and penalties consistent with this Constitution’s Articles, By-Laws and/or Rules and Procedures.

\subsection{Section 2: Managers and Coaches}
Managers and coaches, as leaders, have the responsibility of setting good example for their youths to follow.  They  shall guide their youths in accordance with Catholic principles and the purpose of the CYC Athletic Association  which is primarily dedicated to the formation of character and leadership.

\section{Article IV: Organization}
\subsection{Section 1: Executive Board}
The Executive Board shall consist of one (1) lay representative from each member athletic association, all active Past Chairpersons, Sport Officials-in-Chief, Chairperson, Vice-Chairperson, Secretary, Treasurer, Sergeant at Arms, Special Advisor(s), Equipment Coordinator, and Webmaster.  Duties of the Executive Board are described in the Special Rules and Procedures, Rule ”H”.

\subsection{Section 2: Inauguration of Officers/Term of Office}
The Chairperson, Vice-Chairperson, Secretary, Treasurer, Sergeant at Arms, Special Advisor(s) term of office shall run from January 1st through December 31st.  

\subsection{Section 3: Executive Officer Positions}
The Chairperson and Vice-Chairperson positions are filled by an individual interested/capable of filling the role. The Chairperson and Vice-Chairperson are voted into office by a majority vote of the Executive Board at the December Executive Board meeting.

The Chairperson is voted into office during an even numbered year, while the Vice-Chairperson is voted into office during an odd numbered year. There are no “term-limits”, however positions will be available and voted on every two years of their respective even/odd years.

\subsection{Section 4: League Coordinator}
The Chairperson shall appoint, from the ranks of the individual parish representatives, League (Page) Coordinators for the sport in season.

\subsection{Section 5: Salary}
No member of the Executive Board shall receive any compensation from the North County District CYC Athletic Association for his/her services as board members.

\subsection{Section 6: Financial Affairs}
The financial affairs of the North County District CYC Athletic Association are the sole responsibility of the organization, with all income and disbursements being processed directly through the officers of the Executive Board.  The District will abide by the financial/procedural requirements as set forth by the Archdiocesan Financial Office.

All invoices and bills issued by the NCCYC to its member parishes shall be sent to the parish representative and the sports association treasurer and shall be due and payable within 30 days of receipt.  The receipt date shall be defined as the date mailed with the US Postal Service or sent by other electronic method to the member parish’s treasurer.  Payment shall be considered received when it is received by the Treasurer.  The parish representative will be copied on all correspondence to assist with ensuring the payment of the bill.

In the event a member parish fails to comply with this requirement, then the parish will not be in good standing with the NCCYC and will not be entitled to certain rights and privileges, specifically, representing NCCYC in the Archdiocesan playoffs and receiving trophies as a league winner.

The Chairman or Treasurer will then promptly send a communication to the parish representative and treasurer for that parish advising them they are not in good standing with the district and they have fifteen days to correct this violation.

In the event the parish fails to cure the violation within the specified time frame the member parish will not be in good standing until bill(s) are paid in full.

Should any portion of the invoice be disputed by said athletic association, the undisputed portion is to be paid in full immediately while the disputed amount/issue is under investigation. 

Exceptions:

The chairman shall grant a waiver to this rule in the following circumstances within his/her discretion:
\begin{itemize}[\textbullet]
    \item The balance due is less than \$100
    \item The parish protests the invoice, has a reasonable basis for its protest, and advises the district in writing of its protest within the 14-day deadline with resolution within 30 days. If the protest is not resolved by the district within this 30-day period, the parish's dispute/claim is honored.
\end{itemize}

The Treasurer will maintain a tracking of when its member parishes are not in good standing.  

\subsection{Section 7: Meetings}
\begin{enumerate}[A.]
    \item Executive Board:  The Executive Board shall meet at least monthly at alternating parishes.  A quorum of six (6)
    Parish Representatives within the District must be in attendance to render a valid decision, but a quorum need 
    not be present to conduct a meeting.
    \item Board of Control:  The Board of Control will be convened upon the request of the Chairperson or at the request 
    of any active Board of Control member with the current District Chairperson as the Board of Control Chairperson.
    \item Screening Meeting:  A meeting held to place the teams in the appropriate leagues.  Emphasis shall be based upon 
    the past season’s record and the consensus of the personnel in attendance.
    \item Roster Review:  A meeting to review rosters for the purpose of verifying the criteria regulatory for compliance
	Rosters are to be marked through for players not in compliance with the criteria specified.
    \item Kick-off Meeting:  A general meeting where team managers and coaches receive their schedules, rosters, cover
    rules and general information (scorecards, manager information, constitution, etc.).
\end{enumerate}

\subsection{Section 8: Executive Board Responsibilities}
The Executive Board shall have general charge of the affairs, funds, policies, and property of the NCCYC.

\subsection{Section 9: Dual Responsibilities}
Dual offices by the Executive board members are strongly discouraged, unless approved by the Chairperson.

\subsection{Section 10: Board of Control}
\begin{enumerate}[A.]
    \item The Board of Control consists of all persons who have served as Chairperson and the current District 
    Chairperson and Vice-Chairperson.
    \item The current District Chairperson is the Chairperson of the Board of Control.
    \item Responsibilities of the Board of Control:
    \begin{enumerate}[1.]
        \item Evaluates and either accepts or rejects nominees for the Chairperson and Vice-Chairperson.
        \item Shall be used as a source of advice or advisors to the current District Chairperson, Vice-Chairperson. 
        \item In issues where the Board of Control feels the current Chairperson or Vice-Chairperson are remiss in their duties, they will investigate and report the findings to the Executive Board.
        \item Responsible for the updating of the District Constitution’s Articles, By-laws and Rules and Procedures.
        \item Obtain permits from surrounding municipalities/school districts for the use of fields and gyms.
        \item Fill a temporary vacancy due to a sudden/unexpected departure of an Executive Officer until the parish fills the vacancy within a 60-day period
    \end{enumerate}
\end{enumerate}

\subsection{Section 11: Procedures}
The procedures of this organization shall be as described in "Robert's Rules of Order".  Where these rules conflict with this Constitution’s Articles, By-Laws and Rules and Procedures, the latter shall govern.
    
\subsection{Section 12: Voting Eligibility}
\begin{enumerate}[1.]
    \item A quorum of parish representatives must be present to present motions and vote.
    \item Said quorum shall be a minimum of five (5) parish representatives. Parish Representative is identified as the person speaking and voting on the behalf of the member parish.
    \item Parish Representative is defined as the person speaking and voting on the behalf of the member parish.
    \item Past Chairpersons in attendance.
    \item Sport OIC, only for the sport in-season.
    \item District chairman, only in the case of tie.
\end{enumerate}

\section{Article V: Membership}
\subsection{Section 1: Organized Divisions}
To determine the player's division for Soccer/Volleyball, use the sport year (the calendar year in which the sport year will be played) and subtract the number in the SV column.  Enter the resultant in the Year of Birth column for that number in the SV column.   Do this for all divisions.  For Basketball, Baseball, and Softball, follow the same process, except use the number in column BBS.  Match the player's Year of Birth with the year entered in the Year of Birth column.  If the player was born on or after July 31st, that is his/her correct division.

\begin{center}
    \begin{tabular}{|c|c|c|c|c|c|}
        \hline
         & & \multicolumn{2}{c|}{Soccer} & \multicolumn{2}{c|}{Basketball} \\
         & & \multicolumn{2}{c|}{Volleyball} & \multicolumn{2}{c|}{Baseball \& Softball} \\
         \hline
         Division & Grade & SV & Year of Birth & BBS & Year of Birth \\
         \hline
         Atom & 3 & 10 & & 11 & \\
         \hline
         Bantam & 4 & 11 & & 12 & \\
         \hline
         Midget & 5 & 12 & & 13 & \\
         \hline
         Crusader & 6 & 13 & & 14 & \\
         \hline
         Cadet & 7 & 14 & & 15 & \\
         \hline
         Intermediate & 8 & 15 & & 16 & \\
         \hline
         Juvenile & 9/10 & 17 & & 18 & \\
         \hline
         Junior & 11/12 & 19 & & 20 & \\
         \hline
    \end{tabular}
\end{center}

NOTE:  If the player’s birthdate exceeds July 31st of the next older birth year, he/she must play in that older division. (Each division has a range of 23 months.)

NOTE:  A player must be in the grade designated for the team on which he/she is participating.  (Exception: Baseball, Softball players who have just completed their senior year of high school may participate in the Junior Division.)

In the Intermediate Division and below, the Executive Board has the right, given supporting documentation, to allow a player who has scholastically advanced to a higher grade, to play in a grade lower, provided that player’s age falls within the requested division.

\section{Article VI: Playing Rules}
\subsection{Section 1: Rulebooks}
The playing rules, the Official Rulebooks, as issued by the Archdiocesan CYC, are based on those published by:
\begin{center}
    \begin{tabular}{l l}
        Soccer & United States Soccer Federation \\
        Basketball & National Federation Official Basketball \\
        Baseball & National Baseball Congress of America \\
        Softball & Amateur Softball Association of America \\
        Volleyball & United States Volleyball Association \\
    \end{tabular}
\end{center}
except as modified by the North County CYC Athletic Association, in supplemental rules published prior to the start of each sport season. (See rule XII for additional information.)

\subsection{Section 2: Severe Weather Alert}
In the event of severe weather, or a warning (siren), an official shall immediately stop said game at the completion of the play in progress.  Play shall be resumed from the point of suspension, providing the game has a bearing on the final league standings. A restart of the contest shall be in accordance with the playing rules for that sport. 

In Baseball and Softball, when the heat index is reported to be 103 degrees (or greater) during the game, by the local television or radio stations, the game times are to be shortened by one-half hour. Start times of succeeding games do not change, therefore creating a larger “break time” for the officials. This may only affect the mid to late day games

\section{Article VII: Protests}
\subsection{Section 1: How and When}
All playing rule protests must be made in writing and filed with the Chairperson, and must be accompanied with a protest fee of \$25.00 by check from the athletic association (NO PERSONAL CHECKS).  Only managers and representatives (coach) designated prior to the game can make an official protest.  This protest must be approved by the Parish Representative or his/her alternate.  The manager/coach protesting an infraction of the playing rules must make an announcement to the appointed official as follows:
\begin{center}
    \begin{tabular}{l l}
        Baseball & Before the next pitch or play (See rulebook for exact details) \\
        Softball & Before the next pitch, legal or illegal (See rulebook for exact details) \\
        Volleyball & At the time of the infraction, before play is resumed \\
        Basketball & At the first stoppage of play after the infraction \\
        Soccer & At the first opportunity when play has stopped \\
    \end{tabular}
\end{center}

In all sports, a brief description of the point of protest must be noted on the scorecard at the time of the alleged infraction.  Failure of the official to notify the opposing manager does not negate a protest.   This does not include eligibility (see Section 3).  If the protest is overruled, the protest fee is forfeited. If the protest is upheld, the protest fee shall be returned. A protest concerning an interpretation of a rule, governing fact of play, must be filed or postmarked no later than 72 hours following the game, (excluding Saturdays, Sundays, and Holidays) along with a \$25.00 check with the District Chairperson or Vice-Chairperson 

\subsection{Section 2: Decision}
On point-of-fact, with play going on, the decision of the official shall be final.  On questions of interpretation of a rule, a protest may be announced, and noted on the scorecard, but the official's decision MUST be accepted while the game is in progress.

\subsection{Section 3: Eligibility}
The eligibility of a player(s) may be protested by letter directly with the Chairperson/ Vice-Chairperson not later than the ninth (9th) day following the alleged violation or mailing a protest letter which must be postmarked no later than the ninth (9) day following the alleged violation, or within 72 hours if the game is the last one scheduled for the alleged violating team.  Should an ineligible/illegal player be accused, the Chairperson will immediately notify the alleged offending team’s Parish Representative. Confirmation of ineligibility will result in the possible forfeiture of any subsequent game(s) in which the said player(s) participate, as well as the forfeiture of prior game(s) played, which may include the offending team’s entire sport schedule.

\newpage
\part{By-Laws}

\newpage
\part{North County Rules and Procedures}
\backmatter

\end{document}