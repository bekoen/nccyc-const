\chapter{By-Laws}
\label{chap:bylaw}
\section{By-Law I: Meetings}
\subsection{Section 1: Protest Hearing}
\begin{enumerate}[A.]
    \item Established by Article VII, Section 4. of the Constitution to hear protests.
    \item A protest will first be reviewed by the District Chairperson or Vice Chairperson to determine if it was filed in accordance with Article VII, Section 1 of the Constitution.  If a protest is filed, they will then contact the Official in Chief to determine that the protest is on a rule infraction and not on a judgment call.
    
    If the facts of the rule(s), the game situation, and the game official’s/manager’s reports are such that the results of a protest are obvious, the Executive Officers, to include the Official in Chief, may rule accordingly, negating the need for a protest.  Said decision may be appealed to a Protest Appeal Board by the offended party.
    \item The Executive Board will serve as the Protest Board. It will convene at its monthly Executive Board meeting and will hear protests in accordance with the following guidelines. A “Special” Executive Board Meeting may be called for by the District Chairperson or Vice Chairperson if a timely expedition is required. Members will be given at least forty-eight (48) hours’ notice.
\end{enumerate}

\subsubsection{Protest Board Guidelines}
\begin{enumerate}[A.]
    \item The Protest Board Chairperson will be the District Chairperson or Vice Chairperson.
    \item Membership

    Each active athletic association in the District will be allowed one (1) member who is allowed to vote.  The Parish Representative from each athletic association has the responsibility to attend.  In his/her place, another active member of their parish may attend.  ONLY with prior written permission given by the District Chairperson, may observers be present.
    \item Non-voting Members 

	The Official in Chief should be in attendance upon the request of the Chairperson.  NOTE:  The OIC may participate with the approval of the Chairperson. 

	The Secretary (or a designated alternate) should be in attendance for the purpose of taking, writing and distributing the meeting minutes.  
    \item Meeting   

	The time and location shall be determined by the monthly Executive Board Meeting rotation as identified in Article VII, Section 1 of the Constitution.  A quorum of four (4) is required.  NOTE:  A Protest Board MUST be called within ten (10) days after the receipt of a valid protest.  A “Special” Executive Board Meeting may be called for by the District Chairperson or Vice Chairperson if a timely expedition is required. Members will be given at least forty-eight (48) hours’ notice.

    \item Order of Business
	
       Chairperson:
       Reads the manager's letter of protest.
	
	Chairperson:
       Reads the official's report

       Officials:  
       Official(s), if requested, shall appear individually.  Members may ask questions, when recognized by the 
       Chairperson, pertaining strictly to the protest.

	Chairperson:
       Has full control of the discussion by setting a time limit on each case or on the discussion by each member. 
       He/she shall also solicit participation from all members

	Chairperson:  
       Shall control and solicit discussion as noted above.

    \item Motions

    Any member, with the exception of the protesting and opposing parishes, may place or second motions.

    Chairperson:  Votes only to break a tie.
    
    NOTE:  Voting options are:  YES, NO, ABSTAIN.
\end{enumerate}

\subsubsection{Appeal}
\begin{enumerate}[A.]
    \item Established by Article VII, Section 4 of the Constitution to hear disciplinary action appeals and protest appeals.
    \item An appeal will first be reviewed by the Chairperson or Vice Chairperson to determine if the appeal was filed in accordance with Article VII, Section 1 of the Constitution.
    \item An appeal of a protest board decision or a disciplinary action must be submitted with the offended athletic association’s approval and be accompanied by an athletic association check in the amount of \$35.00.
    \item A second and final appeal may be made to the Archdiocesan CYC Executive Sports Committee and be accompanied by an athletic association check in the amount of \$50.00.
    \item The Executive Board will convene at its monthly Executive Board meeting and function in accordance with the following guidelines.
\end{enumerate}

\subsubsection{Disciplinary Action Guidelines}
All serious misconduct offences and disciplinary actions are identified as:

\begin{center}
    \begin{tabular}{| m{7cm} | m{4cm} |}
        \hline
        Obscene language or gesture toward an opponent, official or spectator & Four Game Suspension \\
        \hline
        Racism/Bias	& One Year Suspension \\
        \hline  
        Fighting (with blows) &	One Year Suspension \\
        \hline
        Player pushing and/or shoving (no blows) & Two Game Suspension \\
        \hline
        Manager, coach, or adult spectator pushing and/or shoving & One Year Suspension \\
        \hline
        Refusing to leave the property after being ejected (park, gym, parking lot, etc.) & One Year Suspension \\
        \hline
        Suspended person present at a came or is in the facility & One Year Suspension \\
        \hline
        Throwing equipment, chair(s), or another object(s) towards an opponent, official or spectator & One Year Suspension \\
        \hline
        Throwing equipment, chair(s), or other object(s) not directed at a person & Four Game Suspension \\
        \hline
        Ejections & Four Game Suspension \\
        \hline
        Confronting an official after being ejected	& One Year Suspension \\ 
        \hline
        Obscene language (not directed towards a person) & Two Game Suspension \\
        \hline
        Other misconduct not defined elsewhere & Up to One Year Suspension as decided by District Executive Board \\
        \hline
    \end{tabular}
\end{center}

Multiple offences are subject to a lifetime suspension as deemed by the District Executive Board		
Any penalty/penalties shall be carried over to the next sport in which they participate.  This shall include 
officiating for any sport.

\subsubsection{Reports}
The District Vice Chairperson shall be responsible for recording all Disciplinary Actions for distribution at the next Executive Board meeting.  This may be a separate handout, or it may be incorporated as part of the Executive Board Minutes.

\section{By-Law II: Registration of Players, Coaches, and Managers}
\subsection{Section 1: Sports}
The North County CYC Athletic Association may provide for, but is not limited to, the following sports:  Baseball, Softball, Soccer, Volleyball, Lacrosse, Chess, Track and Field, Golf and Basketball.

\subsection{Section 2: Team Entry Form}
To enter a team in a league of the Athletic Association, each Team Entry Form, supplied by the District, must be filled out completely, containing the following information:
\begin{enumerate}[A.]
    \item Name of the parish Athletic Association
    \item The name, email address, street address and telephone number of the manager
    \item The division in which the team will participate
    \item League record from previous year, if applicable
    \item Team ranking within its own parish, in each division
    \item Request for special consideration, concerning playing days/times
    \item Request for placement within a division, A1, A2, B1, B2, etc.
    \item Identify Team as Open or Closed.  Should the parish fail to do so, the District shall declare the team to be Open.
    \item Have signature of Parish Sport Coordinator or Parish Representative
\end{enumerate}

\subsection{Section 3: Parish Activity List}
These activities must be parish-wide, a religious event or an academic function.  The District will schedule around these events when submitted.

We will ATTEMPT to schedule around Girl/Boy scout events if they are listed.  We WILL NOT RESCHEDULE around Girl/Boy Scout events if they are not listed.  An exception will be any Archdiocesan Directive. 
One copy should be submitted with the Team Entry Forms.

Below are examples of events that should be included on the Parish Activity List:

\begin{center}
    \begin{tabular}{>{\small}l >{\small}l >{\small}l >{\small}l}
        PSR & Speech Meets & Talent Shows & High School Open House \\
        First Communion & Graduation & D.A.R.E. Graduation & High School Testing \\
        Field Trips	& Plays & Confirmation & Picnic \\
        Retreats & School Camps & Missions & Festival \\
        Track Meets	& School Concerts & Father/Daughter Dance & Carnival \\
        Spelling Bee & Rehearsals & Math Bee & “Lock In” \\
        Geography Bee & Parish Socials & Mother/Daughter Dance	\\
        Dinner	Auctions & May Crowning & & \\
    \end{tabular}
\end{center}

\subsection{Section 4: Rosters}
\begin{enumerate}[A.]
    \item All players’ names (the complete first name, NO nickname, and then the last name), addresses, zip codes, their Parish of Registration, Parish of Residence, School Attending, the day (M-Monday, T-Tuesday, W-Wednesday, Th-Thursday, F-Friday) that they attend PSR, if applicable, the grade that they are currently in and the School District in which they reside, shall be entered on the roster form accordingly.  If a player is being released from another parish, the Type of Release (P – Parish, H – Handicap, OP – Open Parish), the name of the Releasing Parish shall be entered thereon.  If the player is an OPEN player, enter an O in the O field.  If the player is Closed, leave this field blank.
    \item Enter the manager’s/coaches’ names, addresses, zip codes, phone numbers, email address, Archdiocesan coach’s number in the appropriate spaces.
    \item Junior and Juvenile teams must have one (1) adult (21 or older) on the roster (manager, coach).  The roster must reflect an ``A'' at the end of the ``Name'' column to indicate that the participant is an adult.
\end{enumerate}

\subsubsection{Notes:}
\begin{enumerate}[1.]
        \item Player’s names must be entered on the roster in alphabetic order by the STREET ADDRESS at which they reside.
        \item For all fields that require a parish/school name or the school district, the corresponding abbreviation as identified in the District’s Directions for Completing the Team Roster Form shall be used.  If you have a school that is not identified in this directive, contact one of the executive officers for appropriate guidance.
        \item NO player shall be allowed to participate in a game/match unless his/her name appears on the official roster with the official North County CYC approval
        \item The roster MUST be present for ALL games/matches.  The manager has until the end of the game/match to produce said document(s).  Failure to do so will result in a forfeiture of that game/match.
        \item Team Names/submittals must be in the proper format; Grade – Gender – Classification (A/B) -  Parish  -  Coach Name – Open/Close
            
        Example – 8-G-A-SF-Johnson-O

        \item Rosters are to be submitted to the District Chairperson with the appropriate amount check for said team fee.
\end{enumerate}
    
\subsubsection{Roster Distribution:}
\begin{tabular}{l p{9cm}}
    Paper & Identified with Official NC CYC approval and given to the team manager at the Sport Kick-Off Meeting. \\
    Electronic & District Executive Board and Archdiocesan CYC Office \\
\end{tabular}
\plainbreak{1}
Three (3) copies of each release must be submitted at the same time the roster, but no later than the date determined/announced by the Chairperson.   One (1) copy of the release will be provided to the releasing athletic association, one (1) copy for the receiving athletic association and one (1) copy for the Vice Chairperson.

\subsection{Section 5: Roster and Release Review}
A committee chaired by the Vice Chairperson, to include members of the Executive Board, will review the team rosters and player releases for discrepancies.  Any player that is found lacking the proper information per Section 4A during this review will be scratched from the roster.  A player may be re-instated, via resubmittal of the roster. Any discrepancies found will be identified to the appropriate Parish Representative and clarified/corrected prior to the Sport Kick-off Meeting. 

Failure to submit corrected/missing information, including the Parish Move Form by the Parish Representative within 10 days after notification of said Parish Representative, will result in forfeiture of the games played until the information is received.

\subsection{Section 6: Changes}
Modifications to the official roster will be accomplished by resubmitting the roster to the Chairperson or Vice Chairperson and returned to the manager BEFORE the player is considered eligible.  The Vice Chairperson, in his/her absence, his/her chosen alternate or the Chairperson, shall record the date and time of receipt.  If the form is mailed, the postmark is considered the filed date and must be forty eight (48) hours earlier than the date of the game in which the new player(s) participate(s).  Additions must be postmarked or in the hands of the NC CYC, forty-eight (48) hours before the game is played.

The cut-off date for adding players and coaches is as follows:

\begin{center}
    \begin{tabular}{l l l}
        Sport & Division & Date \\
        \hline
        Soccer/Volleyball & All Divisions &	October 1 \\
        Basketball & All Divisions & February 1 \\
        Baseball/Softball & Intermediate and down & May 1 \\
        & Juvenile and up &	July 1 \\
    \end{tabular}
\end{center}
	 
These dates will be published in the Manager Information Sheets for each sport season.  Players who are not active team members should be deleted.  Deletions will be accepted at anytime.

Players and coaches may be added to a team after the designated cut-off date only if there is sufficient reason to justify such action (i.e., loss of person due to injury, person moved out of the parish, etc).  New residence may be added at any time.  The acceptability of such additions will be determined by the District Chairperson.

\section{By-Law III: Eligibility}
\subsection{Section 1: Player}
\subsubsection{A. Atom, Bantam, Midget, Crusader, Cadet and Intermediate Divisions}
A player is eligible to participate in the North County CYC District Program in the above divisions in the following manner:

\begin{enumerate}[1.]
    \item The Parish of Registration or the Parish School that the player attends.  The choice is entirely up to the player.
    \item Parish of Residence  -  The Parish of Residence will apply if Parish of Registration/School Attending does not have a team in the player's division.
    
    A player who changes residence will be eligible to play for his/her previous parish for one (1) year from the date of the move.  A Parish Move Form must be filed with the District Chairperson.  (Submit with the Team Roster)

    \item Notification of a change in Parish of Registration, but not residence must be provided prior to the start of the sport season
\end{enumerate}

\subsubsection{B. Intermediate Division and Down}
If there is no program in the player's athletic association, the player may apply to their Parish Representative for assignment to another athletic association. The official District Release Form must be completed and submitted with the team Roster. The player must be assigned to a team in his/her division, or higher, based on the needs of the athletic association team and not on geographic or other considerations.  Assignment of these players is based upon the approval of the accepting athletic association.  The player(s) is/are subject to the rules of that athletic association. The purpose of this is to aid athletic associations that have a problem getting enough players and also to provide a place for players who wish to participate.

\subsubsection{C. Junior and Juvenile Divisions}
OPEN PLAY ONLY TEAMS are allowed to carry players on their rosters regardless of their participation in other programs with the exception of the Missouri State High School Activities Association (MSHSAA).  Players participating on their high school team are ineligible to play the same CYC sport at the same time.

\subsubsection{D. Excess Players}
If a parish has an excess of players, too many for one (1) team, not enough for two (2), the subject athletic associations, the Parish Representative may attempt to place the excess player(s) on another team up until the cut-off date to acquire players. After that, it will be up to the District Chairperson.  Player assignment will be in accordance with (B) above.

For  “A” Teams:  In accordance with the Archdiocesan Constitution and By-Laws, an excess of players shall be considered to be over:
\begin{center}
    \begin{tabular}{l r}
        \hline
        Soccer & 14 \\
        Volleyball & 9 \\
        Basketball & 8 \\
        Baseball & 12 \\
        Softball & 13 \\
        \hline
    \end{tabular}
\end{center}

For “B” Teams:  The minimum number shall be at the option of the releasing athletic association.  Under no circumstances will the athletic association be allowed to release a player(s) if it brings the team below the maximum allowed to start the game/match.
\begin{center}
    \begin{tabular}{l r}
        \hline
        Soccer & 11 \\
        Volleyball & 6 \\
        Basketball & 5 \\
        Baseball & 9 \\
        Softball & 10 \\
        \hline
    \end{tabular}
\end{center}

The released player(s) must follow the prescribed requirements for obtaining a release.

\subsubsection{E. Player Releases}
A player may play for another athletic association even though their own athletic association has a team in that division. 

\begin{enumerate}[1.]
    \item Releases are required for all out-of-district players, Intermediate and down.
    \item Out of district releases are allowed only with the permission of the Chairman from the releasing District.
    \item In the Juvenile and Junior Divisions, releases are not required for players participating in district games.  However, in order for any of these players to participate in the Archdiocesan Championship Playoffs, releases must be presented to the District Chairperson (Required releases should be turned in with rosters.)
    \item Releases must be approved/signed by the Parish Sport Coordi­nator, Parish Representative and by the Manager(s) of any/all teams the releasing athletic association has in that division.
    \item A team that has released players to another athletic association may not accept players re­leased by another athletic association, and vice versa, (Intermediate and below) unless there are extenuating circumstances in the athletic association. This will be reviewed by the District Chairperson.
    
    If a player has been previously, legally released by his/her athletic association for a given sport, he/she will be considered eligible for release, in the subsequent sport season, to the team to which he/she was originally released to.

    \item A Closed player can be reassigned to a Closed or an Open team, regardless of the Releasing team’s classification.
    
    An Open player can only be reassigned to an Open team, regardless of what the Releasing team’s classification is. That player’s assignment can only be to a team within their district, unless there is no Open team in that district.  If there is no Open team on which the player can participate, then the Open player may cross district lines with the written consent of both District Chairpersons.
\end{enumerate}

\subsubsection{F. Limitations}
For all divisions a player may participate, simultaneously, with one CYC team (Exception: Parochial division) and one or more non-CYC teams, in the same sport, provided their North County CYC team has been declared an Open Team and is approved by their Parish Athletic Association at the Sport Screening Meeting.  All other teams will be considered Closed and any player found to be participating simultaneously with a team in the North County CYC Athletic Association and any other team in the same sport will be subject to disciplinary action by the Executive Board (EXCEPTION - Junior, Juvenile, Bantam, and Atom are considered Open.)

\subsubsection{G. Compensation}
No player shall receive any monetary compensation for his/her services rendered to a team of the North County CYC. This rule shall not prevent suitable prizes from being given to players. 

\subsubsection{H. Emergency Coach}
In the absence of the official manager/coaches, a substitute manager/coach, with or without a CYC ID Card, NOT shown on the roster, may participate in said game/match as an emergency coach. Emergency coach must have taken the Protecting God’s Children class.

\subsubsection{I. Accomodations}
The North County CYC recognizes that some children may not be capable of playing sports with children their own age.  When an athletic association can document that a handicap exists, a child may be granted permission, on a case by case basis, to play with an age group one (1) year younger.  A form must be completed by the parents and the Parish Athlet­ic Association Chairperson and submitted to the Executive Board for approval before the child may be entered onto the roster.  The player shall be placed on a B-team, if available.

\subsection{Section 2: Identification Cards}
\begin{enumerate}[A.]
    \item All managers, coaches and players must have an Official CYC Identification Card to be eligible to participate in any contest.  Non-playing manager’s/coach’s ID Cards do not expire.  Managers and coaches must have a coaches ID number as provided by Archdiocesan office.  All managers, coaches must have completed the Protecting God’s Children program, police background check, read The Code of Ethical Conduct book, sign the commitment to Ethical Conduct page.  Scorekeepers do require an ID Card if sitting on the bench.
    \item In special cases, the Parish Representative, or alternate, may request from the Chairperson, a temporary permit to allow the player, manager or coach to participate in a contest.  This permit must be requested at least two (2) days prior to any scheduled/rescheduled contest.  This permit will be good for ten (10) days, and no second permit will be issued to anyone, except under extenuating circumstances, at the discre­tion of the District Chairperson.  Temporary permits are NOT valid for the Archdiocesan Playoffs.
    \item The game officials (or District Official) will check the ID Cards and the official roster for each team A specific player card/ID check may be checked against the player if requested by either manager. Spot checks throughout the sport season may happen as well.
    \item If one (1) or more (to include the entire team) participants do not have their ID Card(s) at the start of the contest, that team will forfeit.
    \item In the event of an ejection per Rule IV, the game official shall obtain the ejected person’s ID card and, as such, it becomes part of his/her report. It remains with the North County CYC Athletic Association until the suspension is resolved.
        
    NOTE:  Failure of the official to secure the ID Card does not void the ejection or automatic suspension.  Alternate IDs (e.g., motor vehicle licenses) shall not be surrendered.
\end{enumerate}

\section{By-Law IV: Conduct}
\subsection{Section 1}
All that is dishonorable, unsportsmanlike and ungentlemanly is particularly and expressly condemned.  The player, coach, manager, spectator found guilty of violating these principles shall be removed from the game at the discretion of the official and be put on report to the North County CYC.  Additional discipline/penalties may be assessed.

\subsection{Section 2}
Any person directly or indirectly approaching a player of another team for the purpose of inducing him/her to leave that team shall be liable to punishment as deemed appropriate by the Executive Board.

\subsection{Section 3}
Rules of conduct to be adhered to by participants in all CYC sports:

\begin{enumerate}[A.]
    \item PROFANITY
    
    Will not be tolerated either on or off the court, field or in the gymnasium while under the jurisdiction of the     CYC.
    
    \item UNSPORTSMANLIKE CONDUCT 
    
    Has no place in the field of sports, particularly by Catholic participants.

    \item RESPECT FOR AUTHORITY
    
    The officials are the representatives of the District, and as such, have the authority on the field, court or           gymnasium until they leave the premises. We must insist that respect be shown to proper authority as a part of essential       training, in accordance with the purpose of the CYC.
    
    \item PUNCTUALITY
    
    Both by members of the teams and officials.  All games should start on time, in order not to cause a delay in    the start of the following game(s).

    \item DANGEROUS PLAY
    
	The idea or concept of any game is to play according to the rules and not to cause bodily harm. All players should be taught and disciplined to play the ball and not the opposing player.

    \item SMOKING, INTOXICATING BEVERAGES, CONTROLLED SUBSTANCES, AND FIRE ARMS AS DEFINED BY MISSOURI LAW
    
	Are prohibited on the bench, on the field, on the court or any playing surface.

    \item Priests and other religious persons, managers, coaches, players, scorekeepers and medical personnel are the only persons permitted on the bench.
\end{enumerate}

\subsection{Section 4}
The manager is responsible for his/her actions as well as those of his/her coaches, players and spectators.  IF YOU DO NOT MEET YOUR RESPONSIBILITIES AND DEGRADE THE CYC PROGRAM, YOU DO NOT BELONG IN THE PROGRAM.  If you have constructive criticism regarding officials, contact your Parish Representative.

\section{By-Law V: Suspensions}
\subsection{Section 1}
Any player, coach, manager or spectator is SUSPENDED INDEFINITELY from ALL ACTIVITIES AND FACILITIES under the jurisdiction of the North County CYC Athletic Association, until he/she serves out their disciplinary action, if he/she is:

\begin{enumerate}[A.]
    \item Ejected from a game (EVEN IF NOT put on report by the game/match official).
    \item Put on misconduct report by a game/match official before, during or after a game.
\end{enumerate}

\subsection{Section 2}
\begin{enumerate}[A.]
    \item Suspensions are issued in terms of games or a period of time, and may overlap into succeeding sport seasons.
    \item Suspensions will cover North County CYC League, Division Playoff and Archdiocesan Playoff games.
    \item All tournaments hosted by athletic associations within the North County District must honor our suspensions.
\end{enumerate}

\subsection{Section 3}
If any player, coach or manager participates in a game while under suspension, or a suspended spectator is present, the team will suffer a forfeit and the offender will be served with disciplinary action.

\subsection{Section 4}
This Association recognizes the suspensions of other organizations.

\section{By-Law VI: Verification of Ages}
\subsection{Section 1}
The Parish Athletic Association shall be responsible for the accuracy of information for all participants on the team roster.

\section{By-Law VII: Elimination Contests}
\subsection{Section 1}
At the end of each sport season, there will be a playoff game involving the win­ner of the ``A'' Leagues, if there was more than one “A” League.  The home team will be determined by the flip of a coin, pre­sided over by the game official(s).  Once a team is determined as home team, it will continue to be the home team should the game be suspended or replayed.   Overtime/extra innings are authorized.

\subsection{Section 2}
The Division Champion shall be the winner of the ``A'' League Playoff.  The first-place winner of each ``A'' League of each division will have an opportunity to represent the District in the Archdiocesan CYC Playoffs. “B” teams are not eligible for playoffs.

\subsection{Section 3}
The following procedure will be used to send two (2) teams, within the same di­vision, to the Archdiocesan Playoffs when a request is received from the Archdiocesan office.

\begin{enumerate}[A.]
    \item If there is only one (1) ``A'' League within the division, the league winner and the sec­ond place team will have an opportunity to represent the District.
    \item If there are two (2) ``A'' Leagues within the division:
    
    The ``A1'' league winner will play a contest against the “A2” league winner. The winner will be given the opportunity to represent the District at the Archdiocesan Playoffs and the looser will be given the opportunity to represent the District in the Wild Card spot (if the wild card spot is available to our District)

    \item If any of the above teams are unable to represent the District in the Archdiocesan Playoffs, the District Chairperson shall select a team to take their place.
\end{enumerate}

\subsection{Section 4: Trophies}
The players on all league winning teams will receive individual trophies.  A minimum of three (3) teams of the same classification is required to be eligible for trophies.

\section{By-Law: VIII: Wagering}
NO wagering of any sort will be tolerated. The violator WILL BE expelled from the Program.

\section{By-Law IX: Officials (Referees/Umpires)}
\subsection{Section 1}
The Official in Chief shall be nominated by an Athletic Association and must be approved by the District Chairperson.  His/her duties and term of office are covered in North County Rule ``E'', Section 4 ``Official In Chief.''

The Official in Chief may serve indefinite terms at the discernment of the District Executive Board. When a vacancy occurs, it shall be filled by a representative of participating athletic association next in line, following the rotation identified below

\begin{center}
    \begin{tabular}{ r l }
        1. & St. Sabina \\
        2. & St. Rose Philippine Duchesne \\
        3. & Blessed Theresa of Calcutta \\
        4. & St. Norbert \\
        5. & Christ, Light of Nations \\
        6. & St. Ferdinand \\
        7. & Our Lady of Guadalupe \\
        8. & St. Martin de Porres \\
        9. & St. Ann \\
        10. & St. Angela Merici \\
        11. & Sacred Heart \\
    \end{tabular}
\end{center}

\begin{enumerate}[A.]
    \item If an athletic association cannot provide a volunteer from within its own organization, it may recruit a volunteer from another athletic association. That volunteer will fill the position on the behalf of the said athletic association, not the athletic association in which he/she normally participates.
    \item The specified athletic association has sixty (60) days after receipt of written notification from the Chairperson to fill the position.
    \item If the specified athletic association fails to fill the position, the athletic association, next in the rotation, will have sixty (60) after receipt of written notification from the Chairperson to fill the position.  This cycle will continue until the position is filled.  Until the position is filled, member(s) of the Executive Board will perform  the necessary duties.
    \item The penalty for non-compliance shall be as follows:  The offending athletic association will not participate in any Archdiocesan Playoffs until the position is filled.
    \item The Executive Board may vote to give a waiver or pass over an athletic association that cannot fill a position.
    \item An athletic association may not hold more than two (2) positions at the same time.  If such a situation arises, that athletic association shall be passed over to the next athletic association specified in the rotation.
\end{enumerate}

\subsection{Section 2}
The game officials are obligated to file a written report, within forty-eight (48) hours with the OIC. The OIC shall forward that report to the District Chairperson for all ejections, misconduct and protests.

\subsection{Section 3}
It is the game official's responsibility to make sure that both managers know that a game is being played under protest.  However, failure to notify the other manager shall not void such protest.

\subsection{Section 4}
The following rules are provided for game officials:

\begin{enumerate}[A.]
    \item Excessive, unnecessary roughness shall result in ejection from the game.
    \item Depending on the circumstances, the official does NOT have to give a warning before ejecting a participant, manager, coach, scorekeeper, medical personnel, or spectator. The “warning” has been issued at the sport kickoff meeting.
    \item If, after a warning, team conduct is not satisfactory after continuing, or attempting to continue the game, the official may suspend play.  Final determination of the game will be decided by the Chairperson following the guidelines covered in the By-Laws under By-Law I.
\end{enumerate}

\section{By-Law X: Forfeiture}
\subsection{Section 1}
Any team that forfeits three (3) consecutive games or a total of four (4) league games in a season, will be dropped from the league.  All games that have been recorded will stand and all remaining games will be recorded as forfeits.  For the purpose of this rule, a forfeit is defined as any game that is not started.

\subsection{Section 2}
In the event both teams fail to appear for a game with a sufficient number of players, both teams will be charged with a chargeable forfeit and under no circumstances will the game be rescheduled.  

NOTE:  Does not apply if the game is officially postponed for some other reason.

\subsection{Section 3}
An assessment of \$50.00 per forfeit shall be billed to the athletic association at the end of the sport season.  All forfeit money will go to the Graham/LePage Scholarship Fund.  If a team notifies the League Coordinator at least seven (7) days before a game is to be forfeited, their Athletic Association will not be billed the \$50.00 forfeit fee.

\section{By-Law XI: Parish Constitution and By-Laws}
Any Athletic Association may draw up a constitution and by-laws for the management of that respective parish, not at variance with this Constitution and By-Laws or the spirit thereof, as inter­preted by the Executive Board.  This constitution and by-laws must be on file with the District Chairperson.

\section{By-Law XII: Supplemental Playing Rules and Procedures}
\label{sec:bylaw-12}
\subsection{Section 1}
The North County CYC Supplemental Rules and an Archdiocesan Rulebook will be available to each team manager online at NCCYC website.

\subsection{Section 2}
Any individual may present a proposed supplemental rule directly to the Executive Board for approval (at least 90 days prior to the start of the season).
 
\section{By-Law XIII: Knowledge of Constitution and By-Laws}
Each manager or coach shall be supplied with, or made available to, a copy of this Constitution, By-Laws and Rules. They shall acquaint their coaches, players/parents and spectators with the contents.  Ignorance shall not be accepted as an excuse for violation of this Con­stitution, By-Laws and Rules.

\section{By-Law XIV: Rescheduling}
\begin{enumerate}[A.]
    \item Games can be rescheduled through the appropriate League Coordinator ONLY with the approval of the District Chairperson, after the requesting team fulfills the following requirements:
    
    The request, resulting from a previously approved activity, MUST be made at least five (5) days or more prior to the scheduled game, and if approved, notification must be given to the District Rescheduler no less than four (4) days prior to the scheduled game.  If this timeframe is not met, the game will not be rescheduled, unless it has a bearing upon a first place standing.

    \item Any event not listed on the Athletic Association Parish Activity List must be approved by the District Chairperson.
    \begin{enumerate}[1.]
        \item The request must be made at least five (5) or more days prior to the scheduled game and if approved, notification must be given to the District Rescheduler no less than four (4) days prior to the rescheduled game. If this timeframe is not followed, the game will not be rescheduled unless it has a bearing upon a first place standing.
        \item It MUST be a RELIGIOUS EVENT or ACADEMIC SCHOOL FUNCTION, affecting a majority of the team’s players.
        
        NOTE:  Not to include any Scouting functions or activities NOT shown on the Athletic Association’s Parish Activity List.
        \item NO  activity will be considered if an official Athletic Association Parish Activity List was NOT submitted by the requesting Athletic Association prior to the last day to add/drop a team as designated by the District Chairperson.
    \end{enumerate}

    \item Parish Representatives and managers should be alert for rescheduled and/or unlisted activities and report them as soon as possible to aid in rescheduling.
    \item A rescheduled game will be subject to the same forfeiture rules as any regularly scheduled game.
    \item A special event, (i.e., Hazelwood School District Day at Six Flags), must be approved by the District Chairperson.
    \item The League Coordinator must give the manager or acting manager a minimum of forty-eight (48) hours notice to play a rescheduled game.
    \begin{enumerate}[1.]
        \item Rescheduled games are planned far enough in advance to give managers the opportunity to take advantage of the seven (7) and five (5) day rules.  (This may not be possible during the last two (2) weeks of any sport season.)
        \item Managers shall acknowledge notification to their League Coordinator (Chairperson or Vice Chairperson if the League Coordinator is not available).  Failure to comply shall not be an acceptable reason for additional rescheduling.
    \end{enumerate}
    \item If a manager notifies the North County CYC of a forfeiture in accordance with A.1 or B.1, and on the day of the rescheduled game, the game is cancelled, the team will receive the game back.
    \item Any rescheduled baseball, softball, volleyball or soccer game that is in conflict with a schedule golf match shall be reviewed by the Golf League Coordinator.  The golf match shall be rescheduled to avoid the conflict.
\end{enumerate}

\section{By-Law XV: Athletic Assocation Mergers}
\subsection{Section 1}
All mergers must be approved by the District Executive Board. The following requirements must be met before the Executive Board considers said merger:
\begin{enumerate}[A.]
    \item Both pastors must agree.
    \item The affected athletic associations must agree and show that both parties will participate in the governing of the program.
    \item A genuine need must be shown.  An athletic association that does not have enough players to en­ter teams in the District program will be deemed to have a genuine need.   However, it must be shown that a concerted effort was made to establish a program within that athletic association.
    \item The entire athletic program must be merged, and players from each athletic association must be integrated on all teams.
    \item Responsibility for payment of all CYC fees must be determined.
    \item Merger applications, in writing, must be received by the Executive Board 90 days prior to the proposed date of the merger.
    \item A merger may be rescinded upon written request by the pastor of either partici­pating athletic association prior to the next sport season.
    \item All district rules and disciplinary action(s) evoked on individual parishes shall apply to the merged athletic associations.
    \item Archdiocesan approval must be obtained for a merger to be valid.
\end{enumerate}

\section{By-Law XVI: Seasonal Sports Merger}
This section is blank.

\section{By-Law XVII: Proviso}
Any matter NOT covered in this Constitution, By-Laws and Rules shall be brought to the North County CYC Executive Board in a timely manner. 
   

